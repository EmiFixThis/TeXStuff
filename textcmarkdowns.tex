
\documentclass{article}
\usepackage{verbatim}
\usepackage[parfill]{parskip}

\begin{document}

\section{Text Commands}

\subsection{Semantic Markup}

Semantic markup commands let you tell the interpreter that you want to emphasize a word or statement without specifying how you want it to effect the way the object looks. 
\vspace{0.25in}
\hrule \vspace{0.25in}

\begin{verbatim}
\emph{semantic markup default is italic} 
\end{verbatim}
\vspace{0.25in}
\hrule \vspace{0.25in}

Looks like: \\
\emph{semantic markup default is italic}
\vspace{0.25in}
\hrule \vspace{0.25in}

\begin{verbatim}
\renewcommand\emph{\textbf}
\emph{semantic markup is now bold}
\end{verbatim}

\vspace{0.25in}
\hrule \vspace{0.25in}

\renewcommand\emph{\textbf}
Looks like: \\
\emph{semantic markup is now bold}
\vspace{0.25in}

\subsection{Visual Markup}

Visual markup can be used alone or to implement semantic markup. 
Visual markup changes the look of the text. 

The look of a text is managed over three axes: \textbf{shape}, 
The following commands are visual:

\begin{verbatim}

\texttt{typewriter text}

\textsc{small caps}

\textit{italic text}

\textbf{bold text}

\textmd{

\end{verbatim}

\end{document}
