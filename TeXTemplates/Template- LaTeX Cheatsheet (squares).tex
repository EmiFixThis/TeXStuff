\documentclass[table,cmyk]{article}
\usepackage[a4paper,margin=1cm,landscape]{geometry}
\usepackage{longtable,array,calc}
\usepackage{xcolor}

\makeatletter
\newcommand\ratio[2]{\strip@pt\dimexpr#1pt/#2\relax}
\newcolumntype{A}[2]
{
        >{\begin{minipage}[t]{#2\linewidth-2\tabcolsep-#1\arrayrulewidth}%
        \vspace{\tabcolsep}}%
        c%
        <{\vspace{\tabcolsep}\end{minipage}}%
}
\makeatother

\pagestyle{empty}

\arrayrulewidth=1pt
\tabcolsep=10pt
\arrayrulecolor{red}

\usepackage{lipsum}
\begin{document}
\begin{longtable}
{
    |A{1.5}{\ratio{30}{100}}% 30%
    |A{1}{\ratio{30}{100}}% 30%
    |A{1.5}{\ratio{40}{100}}% 40%
    |%
}\hline
%Cell 1,1
The fundamental theorem of calculus,
\[
\int_a^b f(x)\,\textrm{d}x=F(b)-F(a)
\]
where \[\frac{\textrm{d}F(x)}{\textrm{d}x}=f(x)\]
&
%Cell 1,2
\LaTeX\ will make you confident!
&
%Cell 1,3
\lipsum[1]
\tabularnewline\hline
%Cell 2,1
The fundamental theorem of calculus,
\[
\int_a^b f(x)\,\textrm{d}x=F(b)-F(a)
\]
where \[\frac{\textrm{d}F(x)}{\textrm{d}x}=f(x)\]
&
%Cell 2,2
\LaTeX\ will make you confident!
&
%Cell 2,3
\lipsum[2]
\tabularnewline\hline
\end{longtable}
\end{document}