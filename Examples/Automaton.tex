\documentclass[letterpaper,landscape,KOMA,calcdimensions,display]{powersem}

\usepackage{ngerman}

\slideframe{none}

\pagestyle{empty}

\renewcommand{\slidetopmargin}{5mm}
\renewcommand{\slidebottommargin}{5mm}

\renewcommand{\slideleftmargin}{5mm}
\renewcommand{\sliderightmargin}{5mm}

\usepackage{amssymb}


\usepackage{pstcol}
\usepackage{pst-node}
\usepackage{multido}
\usepackage{array}
\usepackage{colortbl}

\usepackage[scaleupmath]{tpslifonts}

\usepackage{hyperref}

\usepackage[lightbackground,colorhighlight,colormath,coloremph]{texpower}

\usepackage{automata}

\usepackage{fixseminar}

\renewcommand{\inputtapename}{input tape}
\renewcommand{\outputtapename}{output tape}
\renewcommand{\statetransname}{state transition graph}

\begin{document}
\begin{slide}

\renewcommand{\automatonname}{parity test}

\simulatedfa*{\linewidth}{\textheight}
{{(-.5,0.5)\accept 1\comma1},{(.5,.5)1\comma0},{(-.5,-.5)0\comma1},{(0.5,-0.5)0\comma0}}
{0,1}
{%
  []1\comma1;0->0\comma1,[]1\comma1;1->1\comma0,[]1\comma0;0->0\comma0,[]1\comma0;1->1\comma1,%
  []0\comma1;0->1\comma1,[]0\comma1;1->0\comma0,[]0\comma0;0->1\comma0,[]0\comma0;1->0\comma1%
  }
{1}{0,0,0,1,0,1}{7}{10}

\newslide

\renewcommand{\automatonname}{one bit serial adder}
  \simulatema*{\linewidth}{\textheight}
  {{(-.5,-.2)s},{(.5,-.2)c}}
  {00,01,10,11}
  {%
    s;00->s;0,s;01->s;1,s;10->s;1,[]s;11->c;0,%
    []c;00->s;1,c;01->c;0,c;10->c;0,c;11->c;1%
    }
  {1}{00,10,11,11,01,00}{6}{10}
  {,,,,,,,,,,,,,}
\end{slide}
\end{document}
\documentclass[letterpaper,landscape,KOMA,calcdimensions,display]{powersem}

\slideframe{none}

\pagestyle{empty}

\renewcommand{\slidetopmargin}{5mm}
\renewcommand{\slidebottommargin}{5mm}

\renewcommand{\slideleftmargin}{5mm}
\renewcommand{\sliderightmargin}{5mm}

\usepackage{amssymb}
\usepackage{wasysym}
\usepackage{stmaryrd}
\usepackage{mathrsfs}
\usepackage{dsfont}

\makeatletter\let\iint\@undefined\let\iiint\@undefined\makeatother
\usepackage[leqno]{amsmath}

\usepackage{amscd}

\usepackage{array}

\usepackage{easymat}

\usepackage[cmbrightmath,scaleupmath]{tpslifonts}

\usepackage[lightbackground,colorhighlight,colormath,coloremph]{texpower}

\usepackage{fixseminar}

\let\name=\textsc

\begin{document}
\begin{slide}
  \begin{minipage}{\linewidth}
  \underl{From The Book.}
  \begin{presentbox}
    \setlength{\abovedisplayskip}{.3\abovedisplayskip}%
    \textbf{(D)}\quad The functions $f$ and $g$ fulfil the same functional equation:
    $f\left(\frac{x}{2}\right)+f\left(\frac{x+1}{2}\right)=2f(x)$ and
    $g\left(\frac{x}{2}\right)+g\left(\frac{x+1}{2}\right)=2g(x)$. 

    For $f(x)$, we obtain this from the addition formulas for the sine and cosine:
    \begin{align*}
      f\left(\textstyle\frac{x}{2}\right)+f\left(\textstyle\frac{x+1}{2}\right)
      &=\pi
      \left[\frac{\cos\frac{\pi x}{2}}{\sin\frac{\pi x}{2}}-\frac{\sin\frac{\pi x}{2}}{\cos\frac{\pi x}{2}}\right]
      \\[2ex]
      &=2\pi\frac{\cos\left(\frac{\pi x}{2}+\frac{\pi x}{2}\right)}{\sin\left(\frac{\pi x}{2}+\frac{\pi x}{2}\right)}
      =2f(x)\text{.}
    \end{align*}

    The functional equation for $g$ follows from
    \begin{displaymath}
      g_N\left(\textstyle\frac{x}{2}\right)+g_N\left(\textstyle\frac{x+1}{2}\right)
      =2g_{2N}(x)+\frac{2}{x+2N+1}\text{.}
    \end{displaymath}
  \end{presentbox}
\end{minipage}%

\newslide

\begin{minipage}{\linewidth}
  \underl{From an undergrad book on calculus.}
  \begin{presentbox}
    \begin{align*}
      c_k&=\frac{1}{2\pi}\int_{0}^{2\pi} f(x) e^{-\mathrm{i}kx}\,\mathrm{d}x
      =\frac{1}{2\pi}\sum_{j=1}^{r}\int_{t_{j-1}}^{t_j} f(x) e^{-\mathrm{i}kx}\,\mathrm{d}x\\
      &=\frac{-\mathrm{i}}{2\pi k}\int_{0}^{2\pi} \varphi(x) e^{-\mathrm{i}kx}\,\mathrm{d}x
      =\frac{-\mathrm{i}\gamma_k}{k}\text{.}
    \end{align*}
    As for all $\alpha,\beta\in\mathds{C}$,
    $\left|\alpha\beta\right|\leq\frac{1}{2}\left(\left|\alpha\right|^2+\left|\beta\right|^2\right)$, it holds that
    \begin{displaymath}
      \left|c_k\right|\leq\frac{1}{2}\left(\frac{1}{\left|k\right|^2}+\left|\gamma_k\right|^2\right)\text{.}
    \end{displaymath}
    From the convergence of $\sum\limits_{k=1}^{\infty}\frac{1}{k^2}$ and
    $\sum\limits_{k=-\infty}^{\infty}\left|\gamma_k\right|^2$, it follows that
    \begin{displaymath}
      \sum_{k=-\infty}^{\infty}\left|c_k\right|<\infty\text{.}
    \end{displaymath}
  \end{presentbox}
\end{minipage}%

\newslide

\begin{minipage}{\linewidth}
  \underl{From an undergrad book on calculus (2nd volume).}
  \begin{presentbox}
    \small
    By \name{Fubini}'s theorem,
    \setcounter{equation}{8}%
    \begin{equation}
      \label{eq:GaussLemma1}
      \int\limits_{Z_\varepsilon}\operatorname{div} F \,\mathrm{d}x 
      = \sum_{k=1}^{n}\,
      \underbrace
      {%
        \int\limits_{Q'}
        \left(
          \int\limits^{h\left(x'\right)-\varepsilon}_{-\infty}\partial_kF_k\left(x',x_n\right)\,\mathrm{d}x_n
        \right)
        \,\mathrm{d}x'
      }_{{}\mathrel{=:} I_k}
      \text{.}
    \end{equation}
    Evaluation of $I_k$: Obviously,
    \begin{displaymath}
      I_n=\int\limits_{Q'}F_n\left(x',h(x'-\varepsilon)\right)\,\mathrm{d}x'\text{.}
    \end{displaymath}
    In the case $1\leq k \leq n-1$, we employ the identity
    \begin{displaymath}
      \partial_k
      \left(
        \int\limits^{h\left(x'\right)-\varepsilon}_{-\infty}\!\!\!\!\!\!F_k\left(x',x_n\right)\,\mathrm{d}x_n
      \right)
      =
      \begin{array}[t]{@{}>{\displaystyle}l@{}}
        \int\limits^{h\left(x'\right)-\varepsilon}_{-\infty}
        \!\!\!\!\!\!\partial_kF_k\left(x',x_n\right)\,\mathrm{d}x_n\\
        {}+F_k\left(x',h(x'-\varepsilon)\right)\cdot\partial_k h\left(x'\right)\text{.}
      \end{array}
    \end{displaymath}
  \end{presentbox}
\end{minipage}%

\newslide

\begin{minipage}{\linewidth}
  \underl{From a book on functional analysis.}
  \begin{presentbox}
    \textbf{Definition 25}\quad Let $\mathcal{C}$ and $\mathcal{D}$ be categories and $\mathcal{F}, \mathcal{G}$
    functors from $\mathcal{C}$ into $\mathcal{D}$. A mapping
    $\eta:\operatorname{Ob}\mathcal{C}\to\operatorname{Mor}\mathcal{D}$ is called a \concept{natural transformation
      between $\mathcal{F}$ and $\mathcal{G}$} if
    \begin{enumerate}
    \item[(i)] $\forall
      A\in\operatorname{Ob}\mathcal{C}:
      \eta(A)\in\operatorname{Mor}_{\mathcal{D}}\left(\mathcal{F}(A),\mathcal{G}(A)\right)$
    \item[(ii)] $\forall A,B\in\operatorname{Ob}\mathcal{C}\;\forall f\in\operatorname{Mor}_{\mathcal{C}}(A,B):$
      \begin{align*}
        \begin{CD}
          \mathcal{F}(A)@>{\mathcal{F}(f)}>>\mathcal{F}(B)\\
          @V{\eta(A)}VV                     @VV{\eta(B)}V\\
          \mathcal{G}(A)@>>{\mathcal{G}(f)}>\mathcal{G}(B)\\
        \end{CD}
        &&\text{or}&&
        \begin{CD}
          \mathcal{F}(A)@<{\mathcal{F}(f)}<<\mathcal{F}(B)\\
          @V{\eta(A)}VV                     @VV{\eta(B)}V\\
          \mathcal{G}(A)@<<{\mathcal{G}(f)}<\mathcal{G}(B)\\
        \end{CD}
      \end{align*}
      respectively, commute, if $\mathcal{F}, \mathcal{G}$ are covariant or contravariant, respectively.
    \end{enumerate}

    This is denoted as $\eta:\mathcal{F}\to \mathcal{G}$. Such a natural transformation is called a \concept{natural
      equivalence between $\mathcal{F}$ and $\mathcal{G}$} if $\eta(A)$ is an isomorphism for every
    $A\in\operatorname{Ob}\mathcal{C}$.
  \end{presentbox}
\end{minipage}%

\newslide

\begin{minipage}{\linewidth}
  \underl{From an undergrad book on linear algebra.}
  \begin{presentbox}
    \textit{Step 2.}\quad Determine an eigenvector $v_2$ for an eigenvalue $\lambda_2$ of $F_2$ ($\lambda_2$ is also
    an eigenvalue of $F_1$). Next, determine a $j_2\in\{1,\dots,n\}$ such that
    \begin{displaymath}
      \mathfrak{B}_3 := (v_1,v_2,w_1,\dots,\widehat{w_{j_1}},\dots,\widehat{w_{j_2}},\dots,w_n)
    \end{displaymath}
    is a base of $V$.

    Next, calculate
    \vspace*{-\baselineskip}
    \begin{displaymath}
      M_{\mathfrak{B}_3}(F)=
      \left(
        \begin{MAT}(b){ccccccc}
          \lambda_1&\cdot&\cdot&\cdot&\cdot&\cdot&\cdot\\
          0&\lambda_2&\cdot&\cdot&\cdot&\cdot&\cdot\\
          \cdot&0&&&&&\\
          \cdot&\cdot&&&&&\\
          \cdot&\cdot&&&A_3&&\\
          \cdot&\cdot&&&&&\\
          0&0&&&&&
          \addpath{(2,0,0)rrrrruuuuulllllddddd}\\
        \end{MAT}
      \right)\text{.}
    \end{displaymath}
    If $W_3:=\operatorname{Span}(w_1,\dots,\widehat{w_{j_1}},\dots,\widehat{w_{j_2}},\dots,w_n)$, then $A_3$
    determines a linear mapping $F_3:W_3\to W_3$.
  \end{presentbox}
\end{minipage}%

\newslide

\begin{minipage}{\linewidth}
  \underl{From an undergrad book on linear algebra (2nd volume).}
  \begin{presentbox}
    \DeclareRobustCommand{\with}{\;\vline\;}%
    \DeclareRobustCommand{\Set}[2]{\left\{#1\with#2\right\}}%
    \setlength{\abovedisplayskip}{.5\abovedisplayskip}%
    \setlength{\belowdisplayskip}{.5\belowdisplayskip}%
    \textit{Remark.}\quad If $\left(Y_i\right)_{i\in I}$ is a family of affine subspaces $Y_i$ of an affine space $X$,
    then 
    \begin{displaymath}
      Y := \bigcup_{i\in I} Y_i\subset X
    \end{displaymath}
    is again an affine subspace. If $Y\neq\emptyset$, then 
    \begin{displaymath}
      T(Y)=\bigcup_{i\in I} T\left(Y_i\right)\text{.}
    \end{displaymath}

    \textit{Proof.}\quad For $Y=\emptyset$, nothing is to be proved. Otherwise, there is a fixed point $p_0\in Y$ such
    that 
    \begin{align*}
      T(Y)&=\Set{\overrightarrow{p_0q}\in T(X)}{q\in\bigcup_{i\in I} Y_i} \\
      &= \bigcup_{i\in I}\Set{\overrightarrow{p_0q}\in T(X)}{q\in Y_i}=\bigcup_{i\in I} T\left(Y_i\right)\text{.}
    \end{align*}
    From this, both claims follow.
  \end{presentbox}
\end{minipage}

\newslide

\begin{minipage}{\linewidth}
  \underl{From a book on measure theory.}
  \begin{presentbox}
    Analogously, the general \concept{associativity} of $\sigma$-Algebra products is shown, that is
    \begin{equation}
      \tag{23.12}
      \left(\bigotimes_{i=1}^{m}\mathscr{A}_i\right)\otimes\left(\bigotimes_{i=m+1}^{n}\mathscr{A}_i\right)
      =\bigotimes_{i=1}^{n}\mathscr{A}_i
      \makebox[0pt][l]{\color{textcolor}\quad($1\leq m<n$).}
      \qquad\qquad\qquad\quad
    \end{equation}
    Statement (23.11) allows to prove the existence of the product measure for all $n\geq 2$ by induction.

    \medskip

    \textbf{23.9 Theorem}\quad\textit{For $\sigma$-finite measures $\mu_1,\dots,\mu_n$ on
      $\mathscr{A}_1,\dots,\mathscr{A}_n$, there exists exactly one measure $\pi$ on
      $\mathscr{A}_1\otimes\dots\otimes\mathscr{A}_n$ such that
      \begin{equation}
        \tag{23.13}
        \pi\left(A_1\times\dots\times A_n\right)=\mu_1(A_1)\cdot\dots\cdot\mu_n(A_n)
      \end{equation}
      for all $A_i\in\mathscr{A}_i$ ($i=1,\dots,n$). Here, $\pi$ is also $\sigma$-finite.}
  \end{presentbox}
\end{minipage}%

\newslide

\begin{minipage}{\linewidth}
  \underl{From a book on probability theory.}
  \begin{presentbox}
    \textbf{17.3 Lemma}\quad\textit{If\/ $T$ takes values exclusively from $\mathds{N}$, then $X_T$ is an
      $\mathscr{F}_T$-measurable random variable with values in $\left(\Omega',\mathscr{A}'\right)$. If only
      $P\left\{T<+\infty\right\}=1$ holds, then up to $P$-almost certain equality there exists exactly one
      $\mathscr{F}_T$-measurable random variable $X^*$ with values in $\left(\Omega',\mathscr{A}'\right)$ which
      fulfils the condition
      \begin{equation}
        \tag{17.7}
        X^*(\omega)=X_{T(\omega)}(\omega)
        \makebox[0pt][l]{\color{textcolor}\quad for all $\omega\in\{T<\infty\}$.}
        \qquad\qquad
      \end{equation}
    }%

    \smallskip

    \textit{Proof.}\quad It suffices to treat the second case and provide an $\mathscr{F}_T$-measurable random
    variable fulfilling the given condition. To this end, choose an arbitrary $\omega'\in\Omega'$. We set
    \begin{displaymath}
      X^*(\omega) :=
      \begin{cases}
        X_{T(\omega)}(\omega),&\omega\in\{T<\infty\}\text{,}\\
        \omega',&\omega\in\{T=\infty\}\text{.}
      \end{cases}
    \end{displaymath}
    For every $A'\in\mathscr{A}'$, it is to be proved that $A := \left\{X^*\in A'\right\}$ is an element of
    $\mathscr{F}_T$. 
  \end{presentbox}
\end{minipage}%

\newslide

\begin{minipage}{\linewidth}
  \underl{From my MSc Thesis.}
  \begin{presentbox}
    \newcommand{\PV}{\operatorname{PV}}%
    If we expand equations (4.102) and (4.103), we get
    \begin{align*}
      \lefteqn{\left(\sum_{q\in\PV}\max\left(M(q),M(\neg q)\right)\right)-\delta}\quad&\\[1ex]
      &=
      \begin{array}[t]{@{}>{\displaystyle}l@{}}
        \sum_{\substack{q\in\PV\\q\neq p}}
        \max
        \left(
          \begin{array}{@{}l@{}}
            \frac{m}{M_{{>}s}'\left(\neg p\right)}\cdot M_{{>}s}'(q)
            +\frac{m}{M_{s}'\left(p\right)}\cdot M_{s}'(q),\\[2ex]
            \frac{m}{M_{{>}s}'\left(\neg p\right)}\cdot M_{{>}s}'(\neg q)
            +\frac{m}{M_{s}'\left(p\right)}\cdot M_{s}'(\neg q)
          \end{array}
        \right)\\[6ex]
        {}-\frac{m}{M_{{>}s}'\left(\neg p\right)}\cdot\delta_{{>}s}'
        -\frac{m}{M_{s}'\left(p\right)}\cdot\delta_{s}'\\[3ex]
        {}-\left(\frac{m}{M_{{>}s}'\left(\neg p\right)}-1\right)\cdot r_1
        -\left(\frac{m}{M_{s}'\left(p\right)}-1\right)\cdot r_2\\[3ex]
        {}-\max(r_1,r_2)+m
      \end{array}
    \end{align*}
  \end{presentbox}
\end{minipage}%

\newslide

\begin{minipage}{\linewidth}
  \underl{From my PhD Thesis.}
  \begin{presentbox}
    \DeclareRobustCommand{\Lcap}{\ensuremath{\sqcap}}
    \DeclareRobustCommand{\FPcapIcup}{\ensuremath{\uplus}}
    \DeclareRobustCommand{\pFl}[1]{\ensuremath{\overline{#1}}}
    \DeclareRobustCommand{\Lprimecup}{\ensuremath{\curlyvee}}
    \def\FpFl(#1,#2)%
    {%
      \ensuremath{\mathord
        {%
          \mathchoice
          {\sideset{^{#1}}{^{\,}}{\mathop{\displaystyle\pFl{#2}}}}%
          {\sideset{^{#1}}{^{\,}}{\mathop{\pFl{#2}}}}%
          {\sideset{^{\scriptscriptstyle#1}}{^{\,}}{\mathop{\scriptstyle\pFl{#2}}}}%
          {\sideset{^{\scriptscriptstyle#1}}{^{\,}}{\mathop{\scriptscriptstyle\pFl{#2}}}}%
        }}%
    }
    \DeclareRobustCommand{\Lprimesub}{\ensuremath{\preccurlyeq}}
    \DeclareRobustCommand{\Lsub}{\ensuremath{\sqsubseteq}}
    \DeclareRobustCommand{\FIsub}{\ensuremath{\subseteqq}}
    By Lemma 2.2.7,
    \begin{displaymath}
      \FpFl(d,a)\FPcapIcup\FpFl(d',b)
      =\FpFl({\left(d\Lprimecup \delta\left(\FpFl(d',b)\right)\right)},{a\Lcap \alpha\left(\FpFl(d',b)\right)}).
    \end{displaymath}
    Furthermore, 
    \begin{align*}
      d&\Lprimesub d\Lprimecup \delta\left(\FpFl(d',b)\right),\\
      a\Lcap \alpha\left(\FpFl(d',b)\right)&\Lsub a.
    \end{align*}
    From this, 
    \begin{displaymath}
      \FpFl(d,a)\FIsub\FpFl(d,a)\FPcapIcup\FpFl(d',b)
    \end{displaymath}
    follows by (2.3).
  \end{presentbox}
\end{minipage}%

\end{slide}
\end{document}
