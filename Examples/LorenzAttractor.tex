%
% imbed javascript animations in PDF files using pdftex
%
% version 0004
% date 20030325
%
% by Jochen Skupin
%
\documentclass[12pt]{scrartcl}

% use times and Mathematica math fonts
\usepackage{wrisym}
% a nice font for the title
\usepackage[sf]{vivaldi}
% rm headers
\renewcommand{\descfont}{\rmfamily}
\renewcommand{\sectfont}{\rmfamily}
\normalfont

% a few other packages
\usepackage{url}
\usepackage{graphicx}

% load the pdfanim style 
% (this style is still in an early test phase and not yet available)
% it is an extension to hyperref.sty and loads it automatically
\usepackage{pdfanim}

\hypersetup{
    pdftitle={The Lorenz Attractor},
    pdfsubject={pdfTeX animation example},
    pdfauthor={Jochen Skupin},
    pdfkeywords={strange lorenz attractor pdf tex animation},
    pdfpagemode=None,
    plainpages=false,
    pdfstartview=Fit,
    breaklinks=true,
    colorlinks=false,
    pdfhighlight=/N,
    % define colors, even if not used
    linkcolor=blue,
    citecolor=blue,
    urlcolor=blue,
    citebordercolor=1 1 1,
    filebordercolor=1 1 1,
    linkbordercolor=1 1 1,
    menubordercolor=1 1 1,
    pagebordercolor=1 1 1,
    urlbordercolor=1 1 1,
    pdfborder=1 1 1
  }

\pagestyle{empty}

\begin{document}
% title
\begin{center}
{\sffamily\Huge The Lorenz Attractor}
\end{center}

\noindent
The Lorenz attractor is a strange attractor that arises in a system of
equations describing the \mbox{2-dimensional} flow of fluid of uniform depth, with
an imposed vertical temperature difference.
In the early 1960s, Lorenz discovered the chaotic behavior of a
simplified \mbox{3-dimensional} system of this problem,
now known as the Lorenz equations:

\begin{eqnarray}
  \frac{\dd}{\dd t}X &=& \sigma(Y-X) \label{eq:X}\\
  \frac{\dd}{\dd t}Y &=& -XZ + rX -Y \label{eq:Y}\\
  \frac{\dd}{\dd t}Z &=& XY - bZ \label{eq:Z}
\end{eqnarray}

\noindent 
The following figure shows the numerical solution of equations 
(\ref{eq:X}) to (\ref{eq:Z}) calculated with
$\sigma = 3$,
$r = 26.5$ and
$b = 1$.
The complex \mbox{3-dimensional} structure of this attractor is
hard to see in a \mbox{2-dimensional} plot. Therefore click on 
the image to start/stop a slow rotation.

\vfill

% load and initialise animation
% arguments:
%   - name of the animation (with this name it can be reused
%     several times in one document)
%   - first part of file name (xxx)
%   - number of animation frames / files (n)
%   - time im ms between two animation frames
% the animation will be composed of files 
% xxx0.pdf, xxx1.pdf ... xxx(n-1).pdf
\PDFAnimLoad{lorenz1}{LorenzPlots/lorenz}{60}{500}

\begin{Form}
  \centering
  % put the animation with name "lorenz1" here
  \PDFAnimation{lorenz1}\\
\end{Form}

\subsection*{Reference}
\begin{itemize}
  \item Lorenz, E. N. "Deterministic Nonperiodic Flow." 
	{\em J. Atmos. Sci.} 20, 130-141, 1963.
\end{itemize}


% add space for tiny comment line
\enlargethispage{5\baselineskip}
% tiny comment line
\vfill\noindent\tiny
This page is a demonstration how to imbed javascript animations 
in PDF files using pdftex.
\hfill Author: Jochen Skupin (email: \url{jochen.skupin@uni-bremen.de})

\end{document}


