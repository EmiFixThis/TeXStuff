\documentclass{article}
\pagestyle{empty}
\usepackage[parfill]{parskip}
\usepackage{minipage}
\usepackage{framed}
\usepackage{geometry}
%\geometry{
%paperheight=68mm,
%paperwidth=91mm,
%total={91mm,68mm},
%left=2mm,
%right=2mm,
%top=2mm,
%bottom=2mm,
%}

\usepackage{minipage}
\usepackage{framed}
\usepackage{amsmath}
\usepackage{amsthm}
\usepackage{amssymb}

\theoremstyle{plain}
\newtheorem{thm}{Theorem}
\newtheorem{pf}{Proof}

\theoremstyle{definition}
\newtheorem{defn}{Definition}

\theoremstyle{remark}\begin{document}
\hfill
\fbox{%
\begin{minipage}[c][6,35cm][c]{7.62cm}
\fbox{%
\begin{minipage}[c][6.1cm][c]{7.35cm}
For $112$:

\[ 112 \leq 2^6 \rightarrow 112 \leq 64 \rightarrow 112 - 64 = 48 \]
\[ 48 \leq 2^5 \rightarrow 48 \leq 32 \rightarrow 48 - 32 = 16 \]
\[ 16 \leq 2^4 \rightarrow 16 = 16 \rightarrow 16 - 16 = 0 \]

\medskip

So we have the decomposition of $112$ is $64, 32, 16$.  

\end{minipage}
}
\end{minipage}
}
\end{document}