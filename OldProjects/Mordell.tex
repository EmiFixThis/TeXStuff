\subsection*{Falting's Theorem (Formerly Mordell Conjecture)}

Conjectured in 1922 by Louis J. Mordell \cite{Mor1922}.

\medskip 

\begin{con} 
If a curve over the field $\mathbb{Q}$ \footnote{Now generalized to any number field.} has genus greater than one then, it has finitely many rational points.
\end{con}

\bigskip

\begin{thm}{Falting's Theorem.} \\
Let $\mathcal{C}$ be an algebraic curve defined over $\mathbb{Q}$, such that $\mathcal{C}$ is non-singular\footnote{A curve which has no singularities; that is, one which has no ill-behaved points such as cusps, blow up points, or points where a curve becomes degenerate.} and has genus $g$. Then one of the following cases shows the 
character of the set of rational points on $\mathcal{C}$: 
\end{thm}

\begin{case}{$g = 0$}
If the genus of $\mathcal{C}$ is zero, then the curve either has no rational points, or infinitely many. The result being that $\mathcal{C}$ can be manipulated in the same way as a conic section.
\end{case}

\smallskip

\begin{case}{$g=1$}
If the genus of $\mathcal{C}$ is one, then $\mathcal{C}$ either has no points, or $\mathcal{C}$ is an elliptic curve and the rational points have the form of a finitely generated abelian group\footnote{The keyword being finite.}.
\end{case}

\smallskip

\begin{case}{$g > 1$}
If the genus of $\mathcal{C}$ is greater than one, then $\mathcal{C}$ has finitely many rational points. 
\end{case}
\cite{Fal1983}
\bigskip 