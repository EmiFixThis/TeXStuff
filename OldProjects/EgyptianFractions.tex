\documentclass{article}
\pagestyle{empty}
\usepackage[parfill]{parskip}
\usepackage{minipage}
\usepackage{framed}
\usepackage{geometry}
%\geometry{
%paperheight=68mm,
%paperwidth=91mm,
%total={91mm,68mm},
%left=2mm,
%right=2mm,
%top=2mm,
%bottom=2mm,
%}

\usepackage{minipage}
\usepackage{framed}
\usepackage{amsmath}
\usepackage{amsthm}
\usepackage{amssymb}

\theoremstyle{plain}
\newtheorem{thm}{Theorem}
\newtheorem{pf}{Proof}

\theoremstyle{definition}
\newtheorem{defn}{Definition}

\theoremstyle{remark}\begin{document}
\hfill
\fbox{%
\begin{minipage}[c][6,35cm][c]{7.62cm}
\fbox{%
\begin{minipage}[c][6.1cm][c]{7.35cm}
\textbf{Egyptian Fraction Sums} \\

\smallskip

\hrule

\medskip

\[ \frac{m}{n} = \frac{1}{a_1} + \frac{1}{a_2} + \cdots + \frac{1}{a_l} \]

\medskip

Where,
\[a_1 \leq a_2 \leq \cdots \leq a_l \]

\end{minipage}
}
\end{minipage}
}
\end{document}