\section{The MacLane and Saunders Families}

It would seem the MacLean family never saw fit to remove all of the stereotypical Scottish temperament once settled in New England, and other close northern states. The temperament seems to have served the family quite well; at least as far as Mac Lane as a mathematician and his father and grandfather as Presbyterian ministers are concerned. Mac Lane's grandfather William served as minister to the Second Presbyterian Church in Stuebenville Ohio until 1883. In that year he had preached favorably of Charles Darwin's theory of evolution for which the church brought him up on charges of heresy. It was made known to him that he would be convicted so he moved his family to New Haven. Connecticut  where he became pastor of the College Street Congregational Church. William eventually earned a doctor of divinity from Yale. 

His father Donald was voted the most eccentric and wittiest member of his class; he also wrote the section of the yearbook which described the junior class. He went on to Yale and upon graduation to the Union Theological Seminary in New York City. Donald loved languages and knew at least 27 different languages he would later write an article on the Lord's Prayer in as many. 

Donald MacLane's neighbor growing up was a parishioner of his father's named George Aretas Saunders the son of a prominent Rhode Island Dentist. Aretas also went to Yale and attended the Sheffield Scientific School as an engineering student. Aretas married Isabelle Andrews and they raised three children one of which was named Winifred. Winifred was talented, as well as beautiful the eldest of the three children. Winifred attended Mount Holyoke a still prominent woman's college where she graduated Phi Beta Kappa. She taught English and Mathematics for a while at the local highschool.

Donald and Winifred had two children in Taftville, Saunders (named for his mother's father) and a daughter Lois. Donald taught his children how to find their way home from inside the nearby forest, he built playhouses, and sledding ramps. Activities Mac Lane would pass on to his own daughters much later. Tragedy struck the young family when Lois died at age 4 of heart failure. Mac Lanes father memorialized his daughter in poetry a gift inherited from his mother at talented poet. Donald was a pacifist, this caused tension between himself and his parishioners when the US entered the first World War in 1917. The family moved to the small village of North Wilbraham, Massachusetts. The Mac Lanes made an addition to their family with a new little brother Gerald born in Springfield in 1918, and a short while another brother David (he would also go on to be a mathematician). Mac Lane's father resigned his parsonage and the family moved to Utica, New York. Saunder's took the required New York Regents Exam and failed the arithmetic portion a fact which had no effect whatsoever on his academic progression and he was passed to the 8th grade. 

Mac Lane was around the age of 12 in Utica and passed the time while not in school with his friend Bill Schmidt. They built a tree house which resided 8 feet above the ground and they had designed themselves, complete with a fireplace. Mac Lane recounts they however forgot that a fireplace required a draft and when they lit their first fire the small house filled with smoke. So they designed and built a second treehouse, this time without any fireplace. Mac Lane earned money by delivering baked goods for his neighbor Mrs. Crippen and bought himself a new sled a Flexible Flyer Racer. He took the sled down a toboggan track and bent a rudder. He was quite disappointed and took the sled to department store where he had purchased it and was very surprised to receive a replacement. He kept the sled his whole life, his daughter Gretchen recounts him methodically icing sled ramps for her carefully measured at a 60 degree slope. 

Mac Lane's father became ill and after a false start at recovery went to a sanitarium. He sent many letters home and Mac Lane and his mother went to visit him once. It would be the last time he would see his father who died a short while after. 