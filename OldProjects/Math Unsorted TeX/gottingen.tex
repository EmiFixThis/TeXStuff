\section{Hilbert's G\"{o}ttingen}

Upon arriving in Germany Mac Lane's first six weeks were spent with a German family in Berlin in order to acclimate himself with the culture of the country. He read a pamphlet outlining current politics and was quite shocked to find that there were at the time 27 different political parties. Once Mac Lane had moved to G\"{o}ttingen he learned of the student culture in particular he found the idea of `color fraternities' amusing (as well as elitist). Although dueling was illegal in Germany the fraternity members were trained to duel with broadswords, many of these students and professors proudly wore fresh scars from these duels on Sunday mornings. This was somewhat encouraged by the ladies who found these scars attractive. Dueling was rare in the math department although popular in the law and medicine departments. Mac Lane was nearly challenged to a duel himself\footnote{The challenge was over a missed snowball in a snowball fight he had joined.} but luckily avoided it by having left his calling card in his room to which the student demanding the duel remarked \emph{`we do not deal with such people.'} which was lucky indeed as George Polya had been expelled from G\"{o}ttingen for declining to duel. 

\textbf{February 12th, 1933} \\
Mac Lane recollects a day he was in Wiemar and had managed a ticket to the Opera house for the 50th Anniversary of Wagner's death. 
\begin{center}
\emph{`In the intermission, I walked out to the lobby. There, twenty-five feet away, stood Hitler and Göring (easy to recognize from their newspaper pictures). At that time (as I did some months later), I did not fully realize the prospects of evil. In later years, I vividly recalled the sight of Hitler, but thought that it took place later, in May 1933. It thus later seemed to me to be the one occasion where (had I carried a weapon) I might have personally changed history.'} \cite{Mac1995}
\end{center}


Mac Lane would be one of the last American mathematician's to see the heydays of Hilbert's G\"{o}ttingen. He would see Hitler's rise to power decimate the faculty, and destroy the center of Logic Hilbert had helped construct which would never return to the heights it had reached by that time. This is a story best told by Mac Lane himself and the reader is encouraged to locate a copy of volume 42, issue number 10 of \emph{The Notices of the AMS}, pages 1134-1138 you will find Mac Lane's own recollection of the events he witnessed in Germany. No doubt his own recollection will be much better than any paraphrasing by a well meaning student. 

\section{Teaching}

\todo{Mac Lane at Harvard}

\todo{Mac Lane at Cornell}

\todo{Mac Lane as a Professor at Chicago}

\section{A Survey of Modern Algebra}

My favorite book before Mac Lane was one of my favorite mathematicians. Mac Lane and Birkhoff published the first edition of \emph{A Survey of Modern Algebra} in 1941. The book approaches algebra almost immediately in terms of commutative diagrams. 