\documentclass{article}
\pagestyle{empty}
\usepackage[parfill]{parskip}
\usepackage{minipage}
\usepackage{framed}
\usepackage{geometry}
%\geometry{
%paperheight=68mm,
%paperwidth=91mm,
%total={91mm,68mm},
%left=2mm,
%right=2mm,
%top=2mm,
%bottom=2mm,
%}

\usepackage{minipage}
\usepackage{framed}
\usepackage{amsmath}
\usepackage{amsthm}
\usepackage{amssymb}

\theoremstyle{plain}
\newtheorem{thm}{Theorem}
\newtheorem{pf}{Proof}

\theoremstyle{definition}
\newtheorem{defn}{Definition}

\theoremstyle{remark}\begin{document}
\hfill
\fbox{%
\begin{minipage}[c][8.7cm][c]{9.7cm}
\fbox{%
\begin{minipage}[c][8.5cm][c]{9.4cm}
\textbf{Defintion} \\

\smallskip

\hrule

\medskip

Let $p_n$ denote the $n$-th prime. 
Let \[S_{n}(k) = \{p_{i_{1}}p_{i_{2}} \cdots p_{i_{k}} : 1 \leq i_1 < i_2 < \cdots < i_k \leq n \} \] 
i.e. the set of $\binom{n}{k}$ products of $k$ distinct primes from among the first $n$ primes. 
Given a set $X = \{x_1, x_2, \ldots x_m \}$ we let 
\[ P(X) = \bigg \{ \sum_{i=1}^{m} \varepsilon_{i} x_i : \varepsilon_{i} \in \{0,1\} \bigg \}\] i.e. the set of all possible subset sums involving elements of $X$. 

Finally, we let \[L_{n}(k) = P(S_{n}(k)) and \omega_{n}(k) = \sum_{s \in S_{n}(k)} s; \] 
note that $\omega_{n}(k)$ is the maximal element of $L_{n}(k)$. 

\end{minipage}
}
\end{minipage}
}
\end{document}