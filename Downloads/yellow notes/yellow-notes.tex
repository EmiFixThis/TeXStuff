%
%
% Author:
% Efraín Soto Apolinar.
% http://www.aprendematematicas.org/
% 
% This document contains the definition of
% two commands to include Yellow notes in 
% the margin of a page.
%

\documentclass[12pt]{article}
%%%<
\usepackage{verbatim}
%%%>

\begin{comment}
:Title: Yellow notes

This document contains the definition of
two commands to include Yellow notes in 
the margin of a page.

\end{comment}

\usepackage{color}
\usepackage{tikz}
\usepackage{calc}

\setlength{\parskip}{0ex}
\setlength{\parindent}{0ex}

\newlength{\yellownotewidth}
\setlength{\yellownotewidth}{2.5cm}
\newlength{\yellownoteheight}
\setlength{\yellownoteheight}{2.5cm}
%   -   -   -   -   -   -   -   -   -   -   -   -
% Yellow note...
%   -   -   -   -   -   -   -   -   -   -   -   -
\newcommand{\yellownote}[1]{
\marginpar{
    \vspace{-0.5\yellownoteheight}
        \begin{center}
        \begin{tikzpicture}
            \draw[white,fill=gray!25,opacity=0.75,shift={(-0.125,-0.125)}] 
                (0,0) rectangle (\yellownotewidth,\yellownoteheight);
            \draw[fill=yellow!35] (0,0) rectangle (\yellownotewidth,\yellownoteheight);
            \draw[opacity=0.45,fill=gray!50] (0.7\yellownotewidth,0) -- 
                (0.9\yellownotewidth,0.45) -- (\yellownotewidth,0.4) -- cycle;
            \node[blue,below] at (0.5\yellownotewidth,\yellownoteheight) {
                \begin{minipage}{\yellownotewidth-1em}
                    \scriptsize\sf#1
                \end{minipage}
            };
        \end{tikzpicture}
        \end{center}
        \vspace{0.5\yellownoteheight}
    }
}

%   -   -   -   -   -   -   -   -   -   -   -   -
% Resizeable - Yellow note...
%   -   -   -   -   -   -   -   -   -   -   -   -
\newcommand{\resizeableyellownote}[3]{
\setlength{\yellownotewidth}{#1cm}
\setlength{\yellownoteheight}{#2cm}
\marginpar{
    \vspace{-0.5\yellownoteheight}
        \begin{center}
        \begin{tikzpicture}
            \draw[white,fill=gray!25,opacity=0.75,shift={(-0.125,-0.125)}] 
                (0,0) rectangle (\yellownotewidth,\yellownoteheight);
            \draw[fill=yellow!35] (0,0) rectangle (\yellownotewidth,\yellownoteheight);
            \draw[opacity=0.45,fill=gray!50] (0.7\yellownotewidth,0) -- 
                (0.9\yellownotewidth,0.45) -- (\yellownotewidth,0.4) -- cycle;
            \node[blue,below] at (0.5\yellownotewidth,\yellownoteheight) {
                \begin{minipage}{\yellownotewidth-1em}
                    \scriptsize\sf#3
                \end{minipage}
            };
        \end{tikzpicture}
        \end{center}
        \vspace{0.5\yellownoteheight}
    }
}
%
% Fonts
%
\usepackage{slantsc}
\usepackage[sc]{mathpazo}
%
%
%
\begin{document}
%
%
%
To include a yellow note into a document insert the 
code:
\begin{verbatim}
\yellownote{
   Message into a yellow note. We just want to see how it 
   will look.
}
\end{verbatim}
and you will see the first yellow note at the 
margin of the page.
\yellownote{
Message into a yellow note. We just want to see how it 
will look.
}
Include more text then...

This text is only a test. This text is only a test. 
This text is only a test. This text is only a test. 
This text is only a test. This text is only a test. 
This text is only a test. This text is only a test. 
This text is only a test. This text is only a test. 
This text is only a test. This text is only a test. 
This text is only a test. This text is only a test. 
This text is only a test. This text is only a test. 

And when neccesary, change the size of the yellow note 
with the following command:
\begin{verbatim}
\resizeableyellownote{2.5}{1.5}{
   Resizeable yellow note.
}
\end{verbatim}
to get the second one... 
\resizeableyellownote{2.5}{1.5}{Resizeable yellow note.}
Each of the arguments are: \verb|{width}{height}| 
in centimeters. You do not need to include the 
unit in the argument. The instruction already 
knows it. 
This text is only a test. This text is only a test. 
This text is only a test. This text is only a test. 
This text is only a test. This text is only a test. 
This text is only a test. This text is only a test. 
This text is only a test. This text is only a test. 
This text is only a test. This text is only a test. 
This text is only a test. This text is only a test. 
This text is only a test. This text is only a test. 

I hope this help you improve your design.

\end{document}
