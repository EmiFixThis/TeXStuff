% Na pasta onde ficam os arquivos do Modelo TCC do Eduardo, encontre
% o arquivo ifc.sty e logo abaixo dos '\usepackages' acrescente os seguinte códigos.

\let\su@ExpandTwoArgs\relax 
\let\IfSubStringInString\relax 
\let\su@IfSubStringInString\relax 

\usepackage[nomain,acronym,xindy,toc]{glossaries}

% No mesmo diretório abra o arquivo TCC-IFC.tex abaixo da linha onde 
% encontra-se o '\data{2015}' acrescente os seguintes códigos

\makeglossaries

\newacronym{RBC}{RBC}{Raciocínio Baseado em Casos} % exemplo de sigla
\newacronym{IA}{IA}{Inteligência Artifial}         % exemplo e sigla

% Feito isso, dentro do bloco '\begin{document}' abaixo do comando '\listoftables' acrescente
\printglossary[title=Lista de Siglas, type=\acronymtype]

% Para usar a sigla no trabalho, analise as seguintes opções, sendo que
% abaixo está representado da sequinte forma \comando => saída
\acrfull{RBC} => Racícionio Baseado em Casos (RBC)  
\acrshort{RBC} => RBC
\acrlong{RBC} => Racíocinio Baseado em Casos

% Depois disso, no terminal, entre na pasta do Modelo TCC do Eduardo
% e execute o seguinte comando
latex TCC-IFC.tex
makeglossaries TCC-IFC.acn
latex TCC-IFC.tex

% E para finalizar vá até a sua IDE favorita (no meu caso TexMaker) e compile o projeto.