\documentclass{article}
\usepackage{times}

% Puts more text on each page
\addtolength{\topmargin}{-.75in}
\addtolength{\textheight}{1.5in}
\addtolength{\oddsidemargin}{-.5in}
\addtolength{\textwidth}{1.0in}

\title{BibTex and Annotated Bibliographies}
\author{Marie desJardins}

\begin{document}

\maketitle

\section{BibTeX}

BibTeX is a program that works in conjunction with LaTeX to produce
inline citations and references.  To use BibTeX, you need to
create a ``.bib'' file that contains your bibliographic entries in 
BibTeX format.  You use {\tt cite} (and variations thereof) to create
inline citations and
\begin{verbatim}
\bibliographystyle{STYLE}
\bibliography{BIBFILE}
\end{verbatim}
to produce the references section, where {\tt STYLE} is the name of a BibTeX
style file (i.e., STYLE.bst exists in your LaTeX search path), and
{\tt BIBFILE.bib} is your BibTeX file.

To produce a formatted LaTeX output file that includes citations and
references, you must run {\tt pdflatex}, then {\tt bibtex}, then {\tt
  pdflatex} {\em twice} to update all of the symbol table references.
For example,
\begin{verbatim}
% pdflatex sample-bib                         ;; processes sample-bib.tex
% pdfbibtex sample-bib                        ;; processes sample-bib.bib
                                              ;; produces sample-bib.bbl
% pdflatex sample-bib
% pdflatex sample-bib
\end{verbatim}
You now have a ``.pdf'' output file that contains your
document and formatted references.

We'll go over the specifics of BibTeX in class.

The best online sources for BibTeX I've found are:
\begin{verbatim}
"Getting to Grips with LaTex: Bibliographics with BibTeX,"
by Andrew Roberts,
http://www.andy-roberts.net/writing/latex/bibliographies

"BibTeX and Bibliography Styles,"
http://amath.colorado.edu/documentation/LaTeX/reference/faq/bibstyles.html
\end{verbatim}

There are some nice back-end tools for managing a database
of bibtex entries, and exporting individual bibtex entries
when needed.  Some of my students use BibDesk or EndNote
for this purpose.  I haven't yet taken the plunge to choose
and learn one of these tools, so can't give you any advice
about them, but you might want to explore on your own.

\section{Annotated Bibliographies}

Your annotated bibliography is due on Monday, February 27.  In order to use
BibTeX to produce this document, you should download the following files
from the course website:
\begin{verbatim}
plain-annote.bst -- annotated bibliography style file or BibTeX 
   (written and distributed online by Harvey J. Greenberg)
unsrt-annote.bst -- an unsorted version
bib-guidelines.tex -- the latex source for this handout
bib-guidelines.bib -- the associated bibtex file
\end{verbatim}

The command
\begin{verbatim}
\nocite{*}
\end{verbatim}
will include all of the references in your BibTeX file.  Using
\begin{verbatim}
\bibliographystyle{plain-annote}
\bibliography{BIBFILE}
\end{verbatim}
will produce a references section in annotated-bibliography format,
using the {\tt annote} field in the BibTeX entries for the
annotations.  (The references section in this handout were produced in
this way.)

I haven't been able to find a BibTeX style file that will let you
group references together and include section headers (as in Zobel's
annotated bibliography on pp.~164--170).  To get the section headers
that you see in the annotated bibliography here, I explicitly listed
all of the papers I wanted to use, in the order I wanted them to
appear.
\begin{verbatim}
\nocite{boutilier97,boutilier01,bacchus96,cohen01}
\end{verbatim}
I used the
{unsrt-annote.bst} BibTeX style file, then edited the {\tt
bib-guidelines.bbl} file (which BibTeX produces) manually, {\em after}
running BibTeX.  Each section header is produced by the line
\begin{verbatim}
\subsection*{SECTION HEADER}
\end{verbatim}
just before the first {\tt \\bibitem} entry for that section.
This isn't a very pretty solution, because every time you change your
BibTeX database and re-run BibTeX, you have to edit the {\tt .bbl}
file again.  But it works.  (Obviously, your section headers should be
more semantically meaningful than mine...)


\nocite{boutilier97,boutilier01,bacchus96,cohen01}

\bibliographystyle{unsrt-annote}
\bibliography{bib-guidelines}

\end{document}
