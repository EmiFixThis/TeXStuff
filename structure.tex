%%%%%%%%%%%%%%%%%%%%%%%%%%%%%%%%%%%%%%%%%
% Mathematics Article Notes
% LaTeX Template
% Version 1.0 (2/23/16)
%
% This template has been downloaded from:
%
% Adapted from Article Notes Template on LaTeXTemplates.com by
% Authors:
% Vel (vel@latextemplates.com)
% Christopher Eliot (christopher.eliot@hofstra.edu)
% Anthony Dardis (anthony.dardis@hofstra.edu)
% http://www.LaTeXTemplates.com
%
%
% Author Mathematics Article Notes:
% K. M. Short (kshort@iastate.edu)
% License:
% LaTeX Project Public License
%
% MathArticleNotes.tex
%
% This work may be distributed and/or modified under the
% conditions of the LaTeX Project Public License, either version 1.3
% of this license or (at your option) any later version.
% The latest version of this license is in
% http://www.latex-project.org/lppl.txt
% and version 1.3 or later is part of all distributions of LaTeX
% version 2005/12/01 or later.
 %
% This work has the LPPL maintenance status `maintained'.
% 
% The Current Maintainer of this work is K. M. Short
%
% This work consists of the files mathArticleNotes.tex and structure.tex
%%%%%%%%%%%%%%%%%%%%%%%%%%%%%%%%%%%%%%%%%

%----------------------------------------------------------------------------------------
%	REQUIRED PACKAGES
%----------------------------------------------------------------------------------------

\usepackage[includeheadfoot,columnsep=2cm, left=1in, right=1in, top=.5in, bottom=.5in]{geometry} % Margins

\usepackage{hyperref}


\usepackage[parfill]{parskip}
\usepackage{tikz}
\usepackage{amsmath}
\usepackage{amsthm}
\usepackage{amssymb}


% Envs made up in theorem, definition, and remark style (most not usually used)
\theoremstyle{plain}
\newtheorem{thm}{Theorem}
\newtheorem{lem}{Lemma}
\newtheorem{cor}{Corollary}
\newtheorem{prop}{Proposition}
\newtheorem{con}{Conjecture}
\newtheorem{crit}{Criterion}
\newtheorem{ass}{Assertion}

\theoremstyle{definition}
\newtheorem{defn}{Definition}
\newtheorem{exmp}{Example}
\newtheorem{xca}[exmp]{Exercise}
\newtheorem{cond}{Condition}
\newtheorem{prob}{Problem}
\newtheorem*{sol}{Solution}
\newtheorem*{asl}{Alternate Solution}
\newtheorem{algo}{Algorithm}
\newtheorem{que}{Question}
\newtheorem{ans}{Answer}
\newtheorem{axi}{Axiom}
\newtheorem{prot}{Property}
\newtheorem{asu}{Assumption}
\newtheorem{hyp}{Hypothesis}

\theoremstyle{remark}
\newtheorem{rem}{Remark}
\newtheorem*{nt}{Note}
\newtheorem*{nota}{Notation}
\newtheorem{cla}{Claim}
\newtheorem{summ}{Summary}
\newtheorem{case}{Case}
\newtheorem{cln}{Conclusion}

%\usepackage[T1]{fontenc} % For international characters
%\usepackage{XCharter} % XCharter as the main font


\bibliographystyle{amsalpha} % Citation style

\usepackage[english]{babel} % Use english by default

%----------------------------------------------------------------------------------------
%	CUSTOM COMMANDS
%----------------------------------------------------------------------------------------

\newcommand{\articletitle}[1]{\renewcommand{\articletitle}{#1}} % Define a command for storing the article title
\newcommand{\articlecitekey}[1]{\renewcommand{\articlecitekey}{#}} % Define a command for storing the cite key
\newcommand{\articlecitation}[1]{\renewcommand{\articlecitation}{#1}} % Define a command for storing the article citation
\newcommand{\doctitle}{\articlecitekey\ --- ``\articletitle''} % Define a command to store the article information as it will appear in the title and header

\newcommand{\dateannotatestarted}[1]{\renewcommand{\dateannotatestarted}{#1}} % Define a command to store the date when notes were first made

\newcommand{\pubdate}[1]{\renewcommand{\pubdate}{#1}} % Define a command to store the date line in the title

\newcommand{\docauthor}[1]{\renewcommand{\docauthor}{#1}} % Define a command for storing the article notes author

% Define a command for the structure of the document title
\newcommand{\printtitle}{
\begin{center}
\textbf{\Large{\doctitle}}

\docdate

\docauthor
\end{center}
}

%----------------------------------------------------------------------------------------
%	STRUCTURE MODIFICATIONS
%----------------------------------------------------------------------------------------

\setlength{\parskip}{3pt} % Slightly increase spacing between paragraphs

% Uncomment to center section titles
%\usepackage{sectsty}
%\sectionfont{\centering}

% Uncomment for Roman numerals for section numbers
\renewcommand\thesection{\Roman{section}}
