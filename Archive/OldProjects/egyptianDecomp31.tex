\documentclass{article}
\pagestyle{empty}
\usepackage[parfill]{parskip}
\usepackage{minipage}
\usepackage{framed}
\usepackage{geometry}
%\geometry{
%paperheight=68mm,
%paperwidth=91mm,
%total={91mm,68mm},
%left=2mm,
%right=2mm,
%top=2mm,
%bottom=2mm,
%}

\usepackage{minipage}
\usepackage{framed}
\usepackage{amsmath}
\usepackage{amsthm}
\usepackage{amssymb}

\theoremstyle{plain}
\newtheorem{thm}{Theorem}
\newtheorem{pf}{Proof}

\theoremstyle{definition}
\newtheorem{defn}{Definition}

\theoremstyle{remark}\begin{document}
\hfill
\fbox{%
\begin{minipage}[c][6,35cm][c]{7.62cm}
\fbox{%
\begin{minipage}[c][6.1cm][c]{7.35cm}
For $31$:

\[31 \leq 2^4 \rightarrow 31 \leq 16 \rightarrow 31-16 = 15\]
\[15 \leq 2^3 \rightarrow 15 \leq 8 \rightarrow 15 - 8 = 7\]
\[7 \leq 2^2 \rightarrow 7 \leq 4 \rightarrow 7-4 = 3\]
\[3 \leq 2^1 \rightarrow 3 - 2 = 1\]
\[1 \leq 2^0 \rightarrow 1 = 1 \rightarrow 1-1=0\]

\medskip

Then the decomposition of $31$ is $16, 8, 4,$ and $1$.

\end{minipage}
}
\end{minipage}
}
\end{document}