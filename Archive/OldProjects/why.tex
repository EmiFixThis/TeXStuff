\section{Why Category Theory?}

Category theory to pure mathematicians is possibly one of the ultimate forms of abstraction and formalism. Categories take collections of mathematical structures and precisely divide all their properties into only two concepts: objects and arrows (morphisms). A category consists of all objects of a certain kind such as the category of sets, where the \emph{objects} are sets and the \emph{arrows} are functions between the sets. But a set is only one type of object and there are many types of categories, there are even classes of categories such as the class of all group like categories. 

The point of category theory is not the elements but the relations between them. 

\begin{description}
\item \textbf{2-categories} are the relations between relations.
\item \textbf{2-morphisms} talk about morphisms between morphisms. 
\item These notions are built up and go on into many different dimensions from $n$-categories to $k$-morphisms.
\end{description}

But the idea behind a category is relatively simple (in theory) to state: 

That is, a category is a collection either of arrows or morphisms which can only be composed if they are next to each other. 


\subsubsection{Yes, but Why?}

At this point you might be thinking \emph{Yes, all this abstraction I am sure is quite amusing to mathematicians, but what is it good for? Why should I ever need something so abstract?} 
This is true that for mathematicians category theory can be approached from many different directions. It can be approached from a foundational perspective, where the categories provide and fulfill the same abstract notions we have come to expect from set theory. Category theory can be thought of as a method of simplifying abstraction ignoring the guts and messy details of proofs that are arduous and drawn out, using their generalized \emph{`containers'} as a means to make this simplification precise. 

But there is real fun to be had in its application! Specifically, it is because category theory is so abstract that it is so genuinely useful and oddly practical. Category theory has the ability to connect areas that we might never associate with each other if we defined them purely in terms of their complicated inner workings and details, If you only ever studied the inside of one particular atom and never concerned yourself with other atoms or even astrophysics you would necessarily miss out on quite a lot. You might never realize that the way particles spin around a nucleus of an atom is very similar to the way planets orbit a sun, or the way moons orbit planets. You might never think there are groups of objects that are similarly structured so that their outer components orbit (fall without hitting the ground, at least most of the time) around a central object. But someone else probably would have a bit of experience in these seemingly unrelated areas and they might think: \emph{`I wonder what other things behave in this way?'} Which is how unification theories are born (and how some die). And so that is one way to think about category theory it is a way of collecting similar things based on how they relate to other similar things on a big scale while ignoring the microscopic scale of details altogether. 

Practical applications? In this case we aren't told to wait for 50 to 100 years for the applications to catch up there have already been a great deal already! Category theory has been used to model and optimize systems, to generalize and formalize algorithms, to develop new types of programming languages like Haskell and Pascalm, in creating typed assembly languages, and many other big idea applications like chess. But there are also many small applications that don't seem like they would fit into all this \emph{`abstract nonsense'} such as spreadsheet applications, text editors, object oriented programming applications, turing machines, quantum computers. Suddenly it seems like there are not many areas left which categories can't do something in. Mathematician's called category theorists have a subset which seem to go about all day finding new application areas to apply them to. We probably cannot count every way a category might be useful or not useful yet. But we have established that they are just plain useful on grand and small scales. 