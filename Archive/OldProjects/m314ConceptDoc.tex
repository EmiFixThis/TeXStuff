% Concepts Documentation for Graph Theory
% Math314
% Spring2016

\documentclass[10pt]{article}
\usepackage{verbatim}
\usepackage[parfill]{parskip}

\begin{document}


\section{Concepts Documentation for Graph Theory (Math314)}
\vspace{0.10in}
\hrule
\vspace{0.25in}
\subsection{Syntax}


\begin{verbatim}
	\NewConcept{swproduct}{
	    name = software product,  
	    Name = Software Product,    
	    namecmd = \product,          
	    symbols = {p},               
	    symbolcmd = \p                 
	}
\end{verbatim}

\vspace{0.25in}
\hrule
\vspace{0.10in}

\pagebreak

\subsection{Options}

\vspace{0.10in}
\hrule
\vspace{0.10in}

\begin{description}
	\item[name]            
    \item The name of the concept key
	\item[Name]            
	\item The capitalized name of the concept, 
	\item for use in the beginning of a sentence 
	\item the first letter will be capitalized
	\item[plural]            
	\item The plural form of the name(s)
	\item[Plural]            
	\item The capitalized, pluralized name of the concept.
	\item[namecmd]       
	\item The `short' command that may be used to typeset the name of this concept; 
	\item this option has to be specified for any command to be defined. 
	\item The $\{=value\}$ part may be omitted to get the default concept key
	\item[symbols]          
	\item A comma-separated list of symbols 
	\item representative of instances of this concept. 
	\item The default value is delimited by curly brackets $\{\}$ (the empty list)
	\item[symbolcmd]     
	\item The `short' command that may be used to typeset a specific symbol
\end{description}
\vspace{0.10in}
\hrule
\pagebreak


\vspace{.25in}




