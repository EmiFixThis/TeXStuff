\section{Introduction}

The video is sepia toned, and a spotlight illuminates a lectern in the auditorium at the University of Wisconsin for the Marden Lecture in Mathematics (1989). Dr. Jay H. Beder is the first to the podium; a stiff, nervous man with shoulders coming less than half a foot above the lectern as he announces the sponsors and the theme of the lecture: \emph{`Mysteries and Marvels of Mathematics'}. Despite this theme Beder seems miles from marveling at much of anything much less mathematics, if anything he seems nauseous and hurriedly introduces the sponsor. With hardly more than a second of transition the reserved Morris Marden has taken the podium from a much relieved Beder. Marden seems very pleased indeed to be introducing his close friend Mac Lane, only a few years his junior, and briefly outlines Mac Lanes various official and honorary degrees in a nostalgic tone. Marden gracefully concedes the lectern to Mac Lane with an earnest smile.

In contrast to the two previous speakers Mac Lane seems to ready to burst as he takes the podium; sporting a striped tie, grey plaid coat, and red suspenders.. He seems a towering figure, smiling from ear to ear, his gray hair showing all the marks of having been recently combed seems intent on standing nearly straight up on the back of his head. He has deigned to use the lectern as an armrest leaning against its side with one leg crossed in front.  

The first transparency illuminates a haphazard depiction of a triangle. He begins by explaining that mathematics has come from simple knowledge of the Pythagorean theorem to being able to show that the theorem is always true. He stops, and smiles at the picture on the screen then explains to the audience he'd created a very detailed slide containing the proof but somehow lost it on the train. Laughing he replaces the slide with another self-proclaimed `sloppy' slide and the audience laughs with him as a squiggled line drawing of four triangles surrounding square is illuminated on the screen. Mac Lane resumes leaning against the side of the podium, and proceeds to give a very general, accessible proof of the Pythagorean theorem in terms of areas of four triangles and an interior square; enthusiastically shouting `bingo!' upon completion of the proof. 

Moving on to the more general statement of Fermat's Last Theorem (at the time of the lecture still unsolved). He places a new transparency and title `Mordell Conjecture' followed by `Gerd Faltings in 1983' illuminated in Mac Lane's own swirling, calligraphic handwriting; not at all expected from the two preceding hastily drawn slides depicting the Pythagorean theorem. Below the title of the conjecture a long list of names of mathematicians whose discoveries have lead to the conjecture. Mac Lane's tone becomes  serious and his stance is no longer relaxed as he emphasizes that Falting's solution of the Mordell conjecture\footnote{The Mordell conjecture is now called Falting's theorem.} used the results of previous mathematicians to come to his conjecture, and indeed many of the other mathematicians had made intermediate conjectures along the way\footnote{The list of mathematician's given by Mac Lane is located in the back matter.}. He explains enough to ensure the audience is aware that each of the Mathematicians French, Soviet, English, and American added necessary pieces which led to a further understanding of what might be really happening behind this deceptively simple looking equation. Satisfied, Mac Lane once again becomes the showman he moves on to describe a genus of curves and how they relate to surfaces, then on to holes, before tying everything together with the statement of Poincare's conjecture in terms of the previously mentioned tori holes.