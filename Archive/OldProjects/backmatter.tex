\section*{Back Matter}

\subsection*{Falting's Theorem (Formerly Mordell Conjecture)}

Conjectured in 1922 by Louis J. Mordell \cite{Mor1922}.

\medskip 

\begin{con} 
If a curve over the field $\mathbb{Q}$ \footnote{Now generalized to any number field.} has genus greater than one then, it has finitely many rational points.
\end{con}

\bigskip

\begin{thm}{Falting's Theorem.} \\
Let $\mathcal{C}$ be an algebraic curve defined over $\mathbb{Q}$, such that $\mathcal{C}$ is non-singular\footnote{A curve which has no singularities; that is, one which has no ill-behaved points such as cusps, blow up points, or points where a curve becomes degenerate.} and has genus $g$. Then one of the following cases shows the 
character of the set of rational points on $\mathcal{C}$: 
\end{thm}

\begin{case}{$g = 0$}
If the genus of $\mathcal{C}$ is zero, then the curve either has no rational points, or infinitely many. The result being that $\mathcal{C}$ can be manipulated in the same way as a conic section.
\end{case}

\smallskip

\begin{case}{$g=1$}
If the genus of $\mathcal{C}$ is one, then $\mathcal{C}$ either has no points, or $\mathcal{C}$ is an elliptic curve and the rational points have the form of a finitely generated abelian group\footnote{The keyword being finite.}.
\end{case}

\smallskip

\begin{case}{$g > 1$}
If the genus of $\mathcal{C}$ is greater than one, then $\mathcal{C}$ has finitely many rational points. 
\end{case}
\cite{Fal1983}
\bigskip 

\subsection*{List: Mordell Conjecture}

The following is the list given by Mac Lane during the Marden Lecture of mathematicians whose results lead to Falting's solution of the Mordell Conjecture. \\

\begin{description}
\item \textbf{France}
\begin{itemize}\itemsep1pt
\item A. Grothendieck
\item A. Neron
\item M. Raymond
\item J. P. Serre
\item A. Weil
\end{itemize}
\item \textbf{U.S.S.R}
\begin{itemize}\itemsep1pt
\item S. J. Arekolov
\item I. I. Manin
\item A. N. Parsin
\item I. R. Shafarevich
\item Ju. G. Zarhin
\end{itemize}
\item \textbf{United States}
\begin{itemize}\itemsep1pt
\item S. Lang
\item J. Tate
\end{itemize}
\item \textbf{Japan}
\begin{itemize}\itemsep1pt
\item K. Kodaira
\end{itemize}
\item \textbf{Germany}
\begin{itemize}\itemsep1pt
\item C. L. Siegel
\end{itemize}
\item \textbf{England}
\begin{itemize}\itemsep1pt
\item B. Birch
\item P. Swinnerton-Dyer
\end{itemize}
\item \textbf{Scotland}
\begin{itemize}\itemsep1pt
\item W. V. D. Hodge
\end{itemize}
\item \textbf{Italy}
\begin{itemize}\itemsep1pt
\item Jacobians
\item I. Barsotti
\end{itemize}
\end{description}


\subsection*{Irish Diaspora}

During the late 1800's and early 1900's there was an influx of Irish immigrants due to the veto of the \emph{Government of Ireland Bill} or the \emph{First Irish Home Rule Bill} of 1886. This took place during the much larger \emph{Irish Diaspora} which reached its peak in the 1840's with an outflow of 8.5 million Irish leaving the islands. Roughly 5 million Irish immigrated to the United States during this period\cite{Gal2000}.  

