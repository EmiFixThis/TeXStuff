\documentclass[10pt]{article}




%% BASIC PACKAGES
\usepackage[letterpaper,margin=1.5in]{geometry}
\usepackage[fulladjust]{marginnote}
\usepackage[colorlinks=false,pdfborder={0 0 0}]{hyperref}
\usepackage{url}
\usepackage{graphicx}
\graphicspath{ {Graphics/} }
\DeclareGraphicsExtensions{.pdf,.png}
\usepackage{float}
\usepackage[parfill]{parskip}
\usepackage{lastpage}
\usepackage{fancyhdr}
\usepackage{parselines}

%% REFERENCE PACKAGES
\usepackage{concepts}
\usepackage{gloss}
\makegloss
\usepackage[refpages]{gloss}
\usepackage{appendix}
\usepackage{stmaryrd}

%% EDITORIAL COMMANDS
\usepackage[colorinlistoftodos,prependcaption,shadow]{todonotes}

%% MATH PACKAGES
\usepackage{amsmaths}
\usepackage{amsthm}
\usepackage{amssymb}
\usepackage{tikz}

%%% COMMANDS
\newcommand{\bb}[1]{\mathbb{#1}}					% USAGE: \bb{Z}
\newcommand{\pgref}[2]{\textit{(found on page #2)}}
\newcommand{\marg}{\marginnote{}[3cm]}
\newcommand{\fix}[1]{\todo[inline,color=red!60,backgroundcolor=yellow!40,bordercolor=red!60]{FIXME: #1}}
\newcommand{\inref}{\todo[size=\large,linecolor=blue!40,color=blue!40,backgroundcolor=green!20,bordercolor=blue!40][4]{\textbf{INSERT REFERENCE: } #4}}
\newcommand{\mfig}[5]{\missingfigure{\textbf{Figure Description: } #5}}
\newcommand{\rewrite}[1]{\todo[inline,color=yellow!40,backgroundcolor=green!20,bordercolor=green!40]{\textbf{REWRITE:} #1}}
\newcommand{\done}[7]{\todo[done,color=green!40]{\textbf{COMPLETED: } #7}}
\newcommand{\lookup}[8]{\todo[color=blue!40,backgroundcolor=blue!20!white,bordercolor=blue!40,inline]{\textbf{Look Up:} #8}}
\newcommand{\aglo}[9]{\todo[inline,color=green!30,backgroundcolor=blue!20!white,bordercolor=green!30]{\textbf{ADD Glossary Entry: } #9}}
\newcommand{\ge}[14]{\gloss[nocite]{*}{#14}}

\newcommand{\version}[1]{\textsc{v.} #1}
%% USAGE
% \addcontentsline{toc}{section}{Unnumbered Section}
% \section*{}



%%% ENVIRONMENTS
\newenvironment{list}
	{\begin{enumerate} \itemsep{0pt}
		\item #
	 \end{enumerate}
	}

\newenvironment{pic}
	{\begin{pic}
	 \begin{figure}[h]
	 \begin{center}
		\includegraphics[scale=.5]{name}
	 \end{center}
	 \caption{\scriptsize{}}
	 \label{fig:}
	 \end{figure}
	 \end{pic}
	 }


\newenvironment{tpic}
	{\begin{tpic}
	 \begin{center}
	 \begin{tikzpicture}
	 \end{tikzpicture}
	 \caption{\scriptsize{}}
	 \label{fig:}
	 \end{center}
	 \end{tpic}
	 }


\usepackage{fancyhdr}
\pagestyle{fancy}
\usepackage{lastpage}
\setlength{\headheight}{25.2pt}
\setlength{\headsep}{0.35in}
\pagestyle{fancy}
\fancyhf{}
\renewcommand{\footrulewidth}{0.5pt}
\renewcommand{\headrulewidth}{0.4pt}
\rhead{\textbf{Graph Theory}}
\lhead{\textit{K. M. Short}}
\chead{\textbf{Homework: 05} \version{7}}
\rfoot{Page \thepage/\pageref{LastPage}}

% Envs made up in theorem, definition, and remark style (most not usually used)
\theoremstyle{plain}
\newtheorem{thm}{Theorem}
\newtheorem{lem}{Lemma}
\newtheorem{cor}{Corollary}
\newtheorem{prop}{Proposition}
\newtheorem{con}{Conjecture}
\newtheorem{crit}{Criterion}
\newtheorem{ass}{Assertion}
\newtheorem*{derv}{Derivation}

\theoremstyle{definition}
\newtheorem{defn}{Definition}
\newtheorem{exmp}{Example}
\newtheorem{xca}{Exercise}
\newtheorem{cond}{Condition}
\newtheorem{prob}{Problem}
\newtheorem*{sol}{Solution}
\newtheorem*{asl}{Alternate Solution}
\newtheorem{algo}{Algorithm}
\newtheorem{que}{Question}
\newtheorem{ans}{Answer}
\newtheorem{axi}{Axiom}
\newtheorem{prot}{Property}
\newtheorem{asu}{Assumption}
\newtheorem{hyp}{Hypothesis}



\theoremstyle{remark}
\newtheorem{rem}{Remark}
\newtheorem*{nt}{Note}
\newtheorem*{nota}{Notation}
\newtheorem{cla}{Claim}
\newtheorem{summ}{Summary}
\newtheorem{cln}{Conclusion}
\newtheorem*{case}{Case}


\newtheoremstyle{indent}
  {1pt}% space before
  {1pt}% space after
  {\addtolength{\@totalleftmargin}{3.5em} \addtolength{\linewidth}{-3.5em} \parshape 1 5.5em \linewidth}% body font
  { }% indent
  {\bfseries}% header font
  {.}% punctuation
  {.5em}% after theorem header
  {}% header specification (empty for default)
\makeatother


\begin{document}

\thispagestyle{empty}
\maketitle

\newpage

\thispagestyle{empty}

%% TOC
\tableofcontents
\hrule
\listoffigures
\listoftables
\listoftodos

\newpage

\pagestyle{fancy}

\pagenumbering{arabic}



\begin{prob}
Show that there is exactly one positive $k$ such that no graph contains exactly $k$ spanning trees.
\end{prob}

\smallskip

\begin{sol}
Let $G$ be a connected graph. \\
\end{sol}

\smallskip


\begin{case}{$k = 1$ : $G$ is a Spanning Tree} \label{case:1} \\
\parindent If $G$ itself is a spanning tree then $k =1$.
\end{case}

\smallskip

\begin{case}{$2 < k$} \label{case:2} \\
\parindent Suppose that $G$ is not itself a spanning tree and $k$ is strictly greater than 2. \\
Let $C_k \subset G$, then $C_k$ is a fundamental cycle. \\
Since removing one edge from the cycle gives a spanning tree, and since there is a \emph{fundamental cycle} for every edge in the spanning tree, there is a one-to-one correspondence between the edges in the spanning tree and the edges which are not in the spanning tree. \\
So there are $k$ spanning trees in $G$.
\end{case}

\smallskip

\begin{case}{$k = 2$} \label{case:3} \\
\parindent The only case left to show is one where $k = 2$. \\
Suppose that $T_1$ and $T_2$ exist and are the only spanning trees of $G$. \\
If we take either one of these trees, say $T_1$ and add an edge to it then it will be a cycle. \\
\\ \\
\color{green}{\textbf{Rewrite: }\textit{Then there must be an edge which is not in $T_1 + e$, however since $T_1$ is now a cycle it must have a length of at least three and so will have at least three spanning trees.}}
\end{case}

\rewrite{Then there must be an edge which is not in $T_1 + e$, however since $T_1$ is now a cycle it must have a length of at least three and so will have at least three spanning trees.}

\pagebreak

\begin{prob}
	\item[(a)] \label{part:a} Find the number of spanning trees in the graph $G$ depicted below.
	\item[(b)] \label{part:b} Find the number of spanning trees in the graph $G_k$ for $k \geq 5$ depicted below.
\end{prob}

\begin{figure}[H]
\begin{center}
\begin{tikzpicture}
\label{fig:1}
\draw[every node/.style={inner sep=1.8pt,fill,circle}]
(0,0) node[label=left:$ $](a){} -- ++(0:1) node[label=above:$ $](b){} -- ++(120:1) node[label=above right:$ $](c){}
--  (a) -- ++(150:1) node[label=below:$ $](x){}    (c) -- ++(150:1) node[label=below:$ $](y){}
(a) -- ++(270:1) node[label=below:$ $](k){}    (b) -- ++(270:1) node[label=below:$ $](l){}
(b) -- ++(30:1) node[label=below:$ $](m){}    (c) -- ++(30:1) node[label=below:$ $](n){}
(x)--(y) (k)--(l)  (m)--(n);
\draw (0.5,-1.5) node {$G$}
;
\end{tikzpicture}
\end{center}

\hskip 4em
\begin{center}
\begin{tikzpicture}
\label{fig:2}
\draw[every node/.style={inner sep=1.8pt,fill,circle}]
(0,0) node[label=left:$ $](a){} -- ++(0:1) node[label=above:$ $](b){} -- ++(120:1) node[label=above right:$ $](c){}
--  (a) -- ++(160:1) node[label=below:$ $](x){}    (c) -- ++(140:1) node[label=below:$ $](y){}
(a) -- ++(260:1) node[label=below:$ $](k){}    (b) -- ++(280:1) node[label=below:$ $](l){}
(b) -- ++(20:1) node[label=below:$ $](m){}    (c) -- ++(40:1) node[label=below:$ $](n){};
\draw[dashed]
(x) to[bend left=80,looseness=1.5] (y) (k)to[bend right=80,looseness=1.5] (l)  (m)to[bend right=80,looseness=1.5] (n);
\draw
(a) ++(0:0.5) ++(270:0.9) node{$C_k$}
(b) ++(120:0.5) ++(30:0.9) node{$C_k$}
(a) ++(60:0.5) ++(150:0.9) node{$C_k$}
;
\draw (2,-1.5) node {$G_k$}
;
\end{tikzpicture}
\end{center}
\end{figure}

\smallskip

\begin{nt}
\indent Note that \ref{part:a} is the case where $k = 4$.
\end{nt}
\\
\begin{nota}
\indent $C_k$ stands for a cycle on $k$ vertices.
\end{nota}

\medskip

\begin{sol}
\end{sol}

\\

\begin{itemize}
\item We skip \ref{part:a} since proving part b will cover its case.
\item There are three cycles $C_{k_{1}}, C_{k_{2}}, C_{k_{3}}$, removing an edge from any of these cycles will not produce a spanning tree since all share edges with $X$ (are not when combined with $X$ fundamental cycles).
\item Let $X$ be the set of edges in the center triangle.
\end{itemize}

\smallskip

\begin{case}{The edge deleted is not in $X$} \label{case:4} \\
There are three ways to remove an edge from $X$. \\
If when breaking each of the cycles the first edge deleted is not in $X$, then we will have to also remove an edge from $X$. \\
The order of each cycle is $k$ so the number of ways we can remove an edge from a cycle is $k - 1$. \\
Since there are three cycles $C_{k_{1}}, C_{k_{2}}, C_{k_{3}}$ we get $3(k - 1)^3$ different spanning trees from this case. \\
\end{case}

\smallskip

\begin{case}{Only one $C_k$ removes an edge from $X$} \label{case:5} \\
There are only three ways to remove an edge from $X$. \\
Then the remaining two cycles have $(k - 1)^2$ ways, ( $(k-1)$ ways each) to choose an edge not in $X$. \\
As in the first case, the remaining two cycles will not produce trees until an additional edge from $X$ has been removed. \\
So we have the total number of spanning trees produced in this case:
\[ 3 \times 2 \times (k - 1)^2 = 6(k -1)^2 \]
\end{case}

\smallskip

\begin{case}{Only one $C_k$ does not remove an edge from $X$} \label{case:6} \\
The cycle which does not remove an edge from $X$ (first) must remove one of its own edges (chosen in $(k-1)$ ways), then one edge from $X$ (3 possible choices). \\
So we get $3(k - 1)$ spanning trees from this cycle. \\
The remaining two cycles choose an edge from $X$ first which produces a tree right away and has been accounted for in \ref{case:5}.
\end{case}

\smallskip

The above cases give the total number of spanning trees as:
\begin{equation} \label{eq:1}
3(k - 1)^2 + 6(k - 1)^2 + 3(k - 1)
\end{equation}

Applying the general formula to \ref{part:a} gives the total number of spanning trees:

\smallskip


\begin{align*}  \label{eq:2}
3 \times (4-1)^3 + 2 \times 3 \times (4-1)^2 + 3 \times (4-1) \\
&= 3(3)^3 + 2 \times 3 \times (3)^2 + 3(3) \\
&= 3^4 + 2(3)^3 + 3^2 \\
&= 144
\end{align*}


\pagebreak


\begin{prob}
Let $T$ and $T'$ be two spanning trees of a connected graph $G$ of order $n$.
Show that there exists a sequence $T = T_0, T_1,\ldots, T_k = T'$ of spanning trees of $G$ such that $T_i$ and $T_{i+1}$ have $n-2$ edges in common for each $i$ with $1\leq i \leq k -1$.
\end{prob}

\medskip

\begin{sol}
\vskip
Let $T$ and $T'$ be finite spanning trees of $G$. \\
Start with $T = T_0$ we add edges to $T$ and try to make it look like $T'$. \\
Suppose we have added edges up to the $i-th$ tree $T_i$. \\
If $T_i = T'$ then we are done, so assume $T_i \neq T'$. \\
If $T_i \neq T'$ then adding an edge to $T_i$ will create a cycle $C$, meaning that there is an edge $e'$ in $C$ which is not an edge in $T'$. \\
Let $T_{i+1} = T_i + e - e'$. \\
Then $T_{i+1}$ has one more edge in common with $T'$ than it does with $T_i$. \\
Since these spanning trees were finite by supposition this process will eventually terminate; \\
and since there is a one-to-one correspondence between edges in a spanning tree and edges not in a spanning tree there will be some (mid)-point where $T' = T_k$ must be true for some $k$.
\end{sol}


\pagebreak



\begin{prob}
Count the number of spanning trees of the depicted graph using Matrix Tree Theorem.
\end{prob}

\medskip

\begin{center}
\begin{tikzpicture}
\draw[every node/.style={inner sep=1.8pt,fill,circle}]
(0,0) node[label=left:$ $](a){} -- ++(45:1) node[label=above:$ $](b){} -- ++(-45:1) node[label=above right:$ $](c){}
(a) -- ++(-45:1) node[label=below:$ $](d){} -- (c)   (a)--(c)
(c) -- ++(0:1) node{}
(a) -- ++(180:1) node{}
;
\end{tikzpicture}
\end{center}



\begin{sol}

\end{sol}
\begin{figure}[H]
\label{fig:2}
\begin{center}
\begin{tikzpicture}
\draw[every node/.style={inner sep=1.8pt,fill,circle}]
(0,0) node[label=above:$v_1$](a){} -- ++(45:1) node[label=above:$v_2$](b){} -- ++(-45:1) node[label=above right:$v_3$](c){}
(a) -- ++(-45:1) node[label=below:$v_5$](d){} -- (c)   (a)--(c)
(c) -- ++(0:1) node[label=below:$v_4$]{}
(a) -- ++(180:1) node[label=below:$v_0$]{}
;
\end{tikzpicture}
\end{center}
\end{figure}

\medskip

\[L =
\begin{bmatrix}
1 & 0 & 0 & 0 & 0 & -1 \\
0 & 2 & 0 & -1 & 0 & -1 \\
0 & 0 & 1 & -1 & 0 & 0 \\
0 & -1 & -1 & 4 & -1 & -1 \\
0 & 0 & 0 & -1 & 2 & -1 \\
-1 & 0 & -1 & -1 & -1 & 4 \\
\end{bmatrix}
\]

The characteristic polynomial of the Laplacian is: \[x^6 - 14x^5 + 70x^4 -152x^3 + 144x^2 - 48x \]
The eigenvalues are: \[ \lambda_1 = 3 + \sqrt{5}, \lambda_2 = 3 + \sqrt{3}, \lambda_3 = 2, \lambda_4 = 3 - \sqrt{5}, \lambda_5 = 3 - \sqrt{3}, \lambda_6 = 0 \]

The number of spanning trees of the graph is equal to $\frac{1}{n} \lambda_1 \ldots \lambda_{n-1}$ where the eigenvalues are the largest.

\[ \frac{1}{6} \times \lambda_1 \times \lambda_2 \times \lambda_3 \times \lambda_4 \times \lambda_5 = \frac{48}{6} = 8 \]

\pagebreak

\begin{prob}
Count the number of spanning trees of $K_n$ using Matrix Tree Theorem.
\end{prob}

\medskip

\begin{sol}
Note: Cayley's theorem is a special case of the Matrix Tree (Kirchhoff's) theorem. \\

There are $n$ vertices in the graph and each is connected to every other vertex in the graph. The vectors span a space with dimension $n - 1$. There is only one eigenvalue equal to zero every other eigenvalue corresponds to $n$. Reducing the matrix by the $ith$-row and $jth$-column gives us a span with dimension $n - 2$. So we have that for $n$ vertices each has $n-1$ neighbors, and in the reduction each has $n-2$.

\end{sol}
\pagebreak


\end{document}
