\subsection{Annotations of E3D6F711-EED3-4F12-9FF7-49AD48CC5266}

``Underpin- ning our discussion are two themes.
The first is that definitions are constructed.''

``The second theme is that definitions matter.
They have been instrumental in changing the char- acter of modern
cryptography, and, I suspect, have the potential to change the character
of other fields as well.''

``This alternative viewpoint is what might be called scientific realism''

highlight {[}page 1{]}: ``Here we say that C is the way it is because
that is the nature or mathemati- cal or physical reality.''

highlight {[}page 1{]}: ``order to have a successful theory involving C,
it pretty much has to be as it is now. If C is shaped by the
disciplinary culture, this happens in a superficial way. In short, C is
discovered, either through reasoning or the exercise of the scientific
methodology.''

highlight {[}page 1{]}: ``Nobody would contest the claim that a concrete protocol or primitive, something like AES or TLS, is constructed.''

highlight {[}page 1{]}: ``What is at issue is whether or not our basic
notions in cryptography things like a one- way function or a secure
encryption scheme whether these are invented or discovered.
Constructionism or scientific realism.''

highlight {[}page 1{]}: ``The thesis here is that all of cryptography's
notions are highly constructed.''

highlight {[}page 1{]}: ``As a consequence, the field can move in very different ways from the way that it has moved.''

highlight {[}page 1{]}: ``First, a definition here is something that
embodies an important concept in a field''

highlight {[}page 1{]}: ``If I say ``let $m = sqrt{x}$'', that's not
the kind of definition I have in mind."

highlight {[}page 1{]}: ``I insist that definitions be mathematically
rigorous.''

highlight {[}page 1{]}: 
If I say: 
"a message authentication code allows the recipient of a message to verify the claimed identity of its sender,'' 
I've given a description, not a definition."

highlight {[}page 1{]}: ``When we say that some thing, C, is constructed, or socially constructed, we are emphasizing that C need not be the way that it currently is. It is not inevitable.''

highlight {[}page 1{]}: ``Those from an engineering background might im-
plicitly assume that all of cryptography is constructed. But there is an alternative viewpoint,''

highlight {[}page 1{]}: ``We speak of a language L being NP-complete if
two conditions hold: L is in the set we call NP, and for any language A
in NP, the language A polynomial-time reduces to L. Said differently, L
is a hardest language in NP.''

highlight {[}page 1{]}: ``You've just seen an example of a definition
It is indeed a wonderful definition. I haven't defined all of its
constituent parts 
-I didn't define NP or tell you what it means for one language to polynomial-time reduce to another. Let me skip over that.''

highlight {[}page 1{]}: ``Why do I say that this is a wonderful
definition?''

highlight {[}page 1{]}: ``One is an a priori assessment of''

highlight {[}page 2{]}: ``the definition's value. We could say that the notion of NP-completeness is good because it is particularly elegant,
simple, or potentially useful.''

highlight {[}page 2{]}: ``We could argue that it captures strong
intuition, or has various nice properties. You could make this kind of argument.''

highlight {[}page 2{]}: ``That is what good definitions in cryptography
manage to do.''

highlight {[}page 2{]}: ``They emerged rather suddenly, around 1982, in a paper of S. Goldwasser and S. Micali. Before this, the ``classical''
approach in cryptography consisted of recognizing some problem, devising some scheme that aimed to solve it, and waiting to see if any interesting attacks emerged"

highlight {[}page 2{]}: ``In the framework they put forward, one does not begin with a protocol; one begins with a definition.''

highlight {[}page 2{]}: ``Once it has been carefully laid out, then one
devises a protocol.''

highlight {[}page 2{]}: ``practice, we usually give proofs that take the
form of a reduction.''

highlight {[}page 2{]}: ``The proof establishes that some protocol $$\mathbf{\pi}$ meets its definition \textbf{D} as long as some other protocol $\mathbf{\phi}$ meets its
definition \textbf{d}. If you're confident that $\mathbf{\phi} is good in sense of definition \textbf{d} you'll have to believe that $\mathbf{\pi}$ is good in the sense of definition \textbf{D}.''

highlight {[}page 2{]}: ``Cryptography went from being an ad hoc set of techniques to a scientifically rich area well connected to complexity theory, rich in mathematics.''

highlight {[}page 2{]}: ``While initially there was minimal impact of
this line of work on cryptographic practice, this has changed. Provable
security has now come to interact synergistically with the classical approach to doing cryptography and has given rise to many practical and
high-assurance techniques.''

highlight {[}page 2{]}: ``If you want to change the character of a field, providing it with ``\emph{the right}'' definitions is probably one of the most effective routes to take."

highlight {[}page 2{]}: ``From what I've laid out, you might infer that
the overarching purpose of definitions is to enable theorems and
proofs. And it is true that definitions are essential for these
activities. But I think that definitions would be important in
cryptography even if we never used them to give a proof. First, definitions lead to attacks.''

highlight {[}page 2{]}: ``When you carefully define the goal you are
after, you can quite often use that understanding to break a protocol
that was supposed to meet its formerly-undefined aim. Second,
definitions are essential for productive discourse. In cryptography, it seems like a lot that is said doesn't make a whole lot of sense.''

highlight {[}page 2{]}: ``I will start with the notion of a pseudorandom generator. The informal goal here is to create bits that look random (uniformly distributed), even if they are not. In trying to give a
definition for this goal it is important not to think in terms of
anticipated solutions. Instead, one tries to understand what it is
that the question means, what it is to generate random- looking bits.''

highlight {[}page 3{]}: ``(A) = Pr{[}A G(\$) ⇒ 1{]} − Pr{[}A \$ ⇒ 1{]}''

highlight {[}page 3{]}: ``And there is such an approach for dealing with
randomness, Kolmogorov complexity, that dates to 1960's. But this is not
the approach that has had much cryptographic significance.''

highlight {[}page 3{]}: ``Instead, what Blum and Micali, and Yao
suggested, in the early 1980's, was that pseudorandomness is a property
not of a string but of a probability distribution.''

highlight {[}page 3{]}: ``The object we are interested in studying, a
pseudorandom generator (PRG), is a function that maps a ``short
seed'' a binary string into a longer string. In other words, a PRG
is a map $G: \{0, 1\} n → \{0, 1\} N$ where $n$ and $N$ are constants."

highlight {[}page 3{]}: ``We want to measure the measure the ``quality''
of a PRG. If you realize a PRG G by $G(S) = N$, for example, this
doesn't seem like a good PRG, even though it does comply with our
syntax."

highlight {[}page 3{]}: ``Let's suppose n = 100 and N = 200
we aim to stretch 100-bit strings to 200-bit ones. If you take a random string
 S ∈ \{0, 1\} 100 that is truly random, and compute Y = G(S) ∈ \{0, 1\} 200 ,
 
you'll now have an induced distribution on 200-bit outputs. There is probability mass on at most 2 100 points of our size- 2 200 space of possible outputs. So it's actually a very sparse subset of the 200-bit
strings that could ever arise as pseudorandom outputs. The idea for our definition of a PRG's quality is to say that it doesn't ac- tually
matter. If you give an adversary a string that is formed by taking a
random 100-bit string S and ap- plying G, or if, instead, you give the adversary a truly random 200-bit string, our poor adversary won't be
able to tell the difference.''

highlight {[}page 3{]}: ``More precisely, an adversary A is imagined to
possess one of two kinds of oracles. 

One possibility: the adversary
hits a button and, in response, a random 100-bit string S is selected
and the adversary is pre- sented the 200-bit G(S). She can hit the
button as many times as she likes, each time a random S being chosen
afresh. Call this ``first'' world."

highlight {[}page 3{]}: ``Alternatively, the adversary hits the button and, in response, gets a random 200-bit string. She can again hit the button as many times as she likes. Call this the ``second'' world."

highlight {[}page 3{]}: ``We understand our PRG G as good if for any reasonable adversary A, its ability to distinguish if it is in the first world or in the second world is small.''

highlight {[}page 3{]}: ``we associate to A and G a real number Adv prg(A) that captures the adversary's ability to distinguish its twopossible worlds''

highlight {[}page 3{]}: ``Adv prg''

highlight {[}page 3{]}: ``G''

highlight {[}page 3{]}: ``This is the probability that A outputs ``1''
if we answer its button-pressing with pseudorandom bits minus the
probability that it outputs ``1'' if we answer its button-pressing with
truly random bits."

highlight {[}page 3{]}: ``Advantage 0 means the adversary does
terribly.''

highlight {[}page 3{]}: ``Advantage near 1 means the adversary does
great.''

highlight {[}page 3{]}: ``We have now given a cryptographic definition.
The definition is a way of associating to a cryptographic primitive (the PRG G) and an adversary (the algo- rithm A) a real number, the number telling us how well the adversary is doing in attacking the primitive.''

highlight {[}page 3{]}: ``traditional approach to drawing this distinction is to say that G is secure if it's computable in polynomial time (in n) and for every probabilistic polynomial time adversary A, the
advantage Adv prg (A) that A gets in attacking G is G G a negligible
function.''

highlight {[}page 3{]}: ``The technical definition for the last term: ε(n) is negligible if it vanishes faster than the inverse of any polynomial: for every c \textgreater{} 0 there exists an N c such that
ε(n) \textless{} n −c for all n ≥ N c .''

highlight {[}page 3{]}: ``above asymptotic approach lets us draw a
rigorous distinction between secure and not secure PRGs. We have, in the
process, quietly changed our notion of a PRG: n and N are no longer
constants but, rather, the PRG should operate on strings of any length
n, returning outputs of length N(n) \textgreater{} n. The value n is now
called the security parameter''

highlight {[}page 3{]}: ``There's an alternative approach for defining
secu- rity. It says: don't try to make binary distinctions. Once we have said how to associate an advantage Adv prg (A) to each A and G, we have defined security.''

highlight {[}page 3{]}: ``t this point one can already state theorems
relating the advantage that a first adversary can obtain at attacking a
first goal to the advantage that a second adversary can obtain in
attacking a second goal.''

highlight {[}page 3{]}: ``This is the concrete-security approach that M.
Bellare and I helped popularize.''

highlight {[}page 3{]}: ``distinction between asymptotic and concrete
security is important, if it's a significant difference''

highlight {[}page 3{]}: ``One answer you can reasonably give is to say
that the difference between asymptotic and concrete security is not significant because, first, asymptotic security definitions and
theorem can almost always be converted into concrete-security ones.''

stamp {[}page 3{]}: ``Example''

highlight {[}page 4{]}: ``ideas of a definition, theorem, or proof
almost always transcend this concrete vs. asymptotic distinction.''

highlight {[}page 4{]}: ``But you can also make the case that the asymptotic vs. concrete definitional choice is quite significant.''

highlight {[}page 4{]}: ``early character of cryptog- raphy was
profoundly influenced by the early choice of an asymptotic approach.''

highlight {[}page 4{]}: ``Because this approach hides ``low-level''
efficiency matters, effectively treat- ing all polynomials, and all
negligible functions, as equivalent, people focused on broad, abstract
relations among cryptographic primitives"

highlight {[}page 4{]}: ``There was also a focus on public-key
cryptography, with shared-key cryptography being essentially
denigrated.''

highlight {[}page 4{]}: ``One reason for this is the need for a security
parameter in the asymptotic approach, something prevalent in public-key
schemes, but not in symmetric schemes.''

highlight {[}page 4{]}: ``questions be- came visible, things that you
simply do not see if you describe everything in terms of asymptotic
security. Symmetric cryptography joined that ranks of topics having
legitimate scientific credentials.''

highlight {[}page 4{]}: ``Definitional choices impact the types of questions that will be asked, and the types of questions that will be rendered invisible.''

highlight {[}page 4{]}: ``cryptography would initially have been asymptotic, because the founders of the field were coming from a
community that had recently mastered the idea of NP-completeness, and
other complexity classes, a tra- dition that was already steeped in
reductions, polynomiality, and asymptotics''

highlight {[}page 4{]}: ``asymptotic and that that would emphasize the
kind of high-level questions that complexity theory had also come to
focus on.''

highlight {[}page 4{]}: ``The question I want to ask here is what a
block- cipher is. In answer, a blockcipher is a function that takes in a key K from some finite set K of possible keys, and it takes in an n-bit plaintext block for some constant n \textgreater{} 1. It produces a
corresponding cipher- text block, again of length n. We require that
each E K (·) = E(K, ·) be a permutation, a one-to-one and onto
function.''

highlight {[}page 4{]}: ``The above is the syntax of a blockcipher; as
with our treatment of PRGs, I've begun without specifying anything about security. There are lots of approaches to trying to define security.''

highlight {[}page 4{]}: ``All of these notions can be built up into
definitions---we can come up with a rigorous Adv-notion for each. But
none of these''

highlight {[}page 4{]}: ``ideas really work to give us a convenient
primitive.''

highlight {[}page 4{]}: ``The winning approach is to capture that a
blockci- pher should behave like a random permutation.''

highlight {[}page 4{]}: ``I will sketch the pseudorandom permutation
(PRP) notion for blockcipher security.''

highlight {[}page 4{]}: ``we imagine an ad- versary dropped into one of
two possible worlds.''

highlight {[}page 4{]}: ``first of these worlds, the adversary, A, is
given access to a box that computes the blockcipher E for a randomly
chosen key K.''

highlight {[}page 4{]}: ``the beginning of the game, a random key K is
selected from the key space K and you give the adversary blackbox access
to the function E K (·).''

highlight {[}page 4{]}: ``he adversary can query what-''

highlight {[}page 5{]}: ``ever plaintext blocks it likes, getting, in
response to each X ∈ \{0, 1\} n , the output Y = E K (X).''

highlight {[}page 5{]}: ``Each query the adversary asks can be based on
the prior outputs it has learned''

highlight {[}page 5{]}: ``the second world, the ad- versary A, in
response to each query X, gets the im- age of a random permutation π
(again from n bits to n bits) applied to X.''

highlight {[}page 5{]}: ``to each new query X ∈ \{0, 1\} n we return a
new, uniformly chosen Y ∈ \{0, 1\} n . If any query is repeated, we
answer as we did before. We measure the adversary's advantage by''

underline {[}page 5{]}: ``We measure the adversary's advantage''

highlight {[}page 5{]}: ``prp''

highlight {[}page 5{]}: ``Adv (A)''

underline {[}page 5{]}: ``This is the probability that the adversary
outputs 1 when we drop it into the first world, minus the prob- ability
that it o''

underline {[}page 5{]}: ``when we drop it into the second world.''

highlight {[}page 5{]}: ``Informally, blockcipher E is secure as long
for every reasonable adversary A---adversaries that don't spend too much
time computing, have de- scription size that's too big, and don't ask
too many queries---the advantage Adv prp (A) is small---it's a E E
number close to zero.''

highlight {[}page 5{]}: ``Even crypt- analysts have come to accept these
notions, no longer viewing key recovery as the one and only property to
violate to have a convincing attack.''

highlight {[}page 5{]}: ``First, as with PRGs, we separated the syntax
of the object from its security notion.''

highlight {[}page 5{]}: ``Second, sim- ple, pessimistic
definitions---meaning that they give the adversary credit quite
generously---are often bet- ter choices than more complex and faithful
ones. Ou''

highlight {[}page 5{]}: ``definition was only a thought experiment for
defining security; in making a definition like this we are not trying
here to faithfully capture an adversary's ca- pabilities in some usage
environment; we are seeking a simple definition that pessimistically
measures the worth of the primitive''

highlight {[}page 5{]}: ``would say that definitions can, in fact, be
wrong.''

highlight {[}page 5{]}: ``earlier of definitional routes not taken are
wrong in the sense that they do not give rise to nearly as useful a
primitive.''

highlight {[}page 5{]}: ``= Pr{[}A E K ⇒ 1{]}''

underline {[}page 5{]}: ``Pr{[}A E K ⇒ 1{]} − Pr{[}A π ⇒ 1{]}.''

highlight {[}page 5{]}: ``− Pr{[}A π ⇒ 1{]}.''

highlight {[}page 5{]}: ``When we formalize what an encryption scheme
is, the approach, going back to Goldwasser and Micali (1982) and then
adapted to the symmetric setting by Bellare, Desai, Jokipii, and me
(1997), is this''

highlight {[}page 5{]}: ``The encryption algorithm takes in a key K and
a plaintext M.''

highlight {[}page 5{]}: ``It produces a ciphertext C.''

highlight {[}page 5{]}: ``The encryption algorithm may be
probabilistic---it can exploit internal ``coins'' (ran- domness) if it
so wishes."

highlight {[}page 5{]}: ``Correspondingly, the ciphertext may be
longer than the ciphertext.''

highlight {[}page 5{]}: ``Decryption must reverse en- cryption: D K (C)
must be M whenever C ← E K (M).''

highlight {[}page 5{]}: ``Our security notion captures an adversary's
inability to distinguish''

highlight {[}page 5{]}: ``the encryptions of equal-length strings.''

highlight {[}page 5{]}: ``question I would like to ask is whether or not
it was necessary to formalize symmetric encryption in roughly this
way.''

highlight {[}page 5{]}: ``Here's an alternative I now favor---it's
usually called authenticated encryption with associated data (AEAD)''

highlight {[}page 5{]}: ``Again focusing on the syntax, an en- cryption
algorithm will now be understood to take in a key K and a message M,
but, also, an initialization vector, IV, and a header, A.''

highlight {[}page 5{]}: ``do so deterministically---no coins allowed. We may assume this time that the length of the ciphertext is the length of the plaintext.''

highlight {[}page 5{]}: ``It takes in the key, the IV, the header, and
the ciphertext, and it produces the plaintext or else a distinguished
symbol ⊥, which is used to indicate that the provided ciphertext does
not correspond to a valid plaintext.''

highlight {[}page 6{]}: ``adversary's inability to distinguish the
encryptions of equal-length strings and, also, its inability to produce
a new ciphertext having a valid associated plaintext.''

highlight {[}page 6{]}: ``First, we do not have to ask our encryption
algorithm to generate good ran- dom bits;''

highlight {[}page 6{]}: ``in fact, we forbid them from generating any
random bits.''

highlight {[}page 6{]}: ``Second, the provisioning of authenticity makes
for a scheme that is easier to correctly use.''

highlight {[}page 6{]}: ``Third, in the absence of an explicit header,
one could not do something as simple as authenti- cate the source
address in a networking packet.''

highlight {[}page 6{]}: ``CCM is the method by which one nowadays
encrypts in WiFi net- works, while GCM is one of the permitted methods
for IPSec.''

highlight {[}page 6{]}: ``First, we have evidenced that questions
utterly basic to a field---like the question ``what is symmetric en-
cryption?''"

highlight {[}page 6{]}: ``Second, I would emphasize that definitions are
not written in stone''

highlight {[}page 6{]}: ``Third, I would conclude that how we define
something---simple things like how many arguments get fed into an
encryption scheme---can have a pro- found effect on how useful that
object will be.''

highlight {[}page 6{]}: ``Smart people can mess up when they don't
under- stand the underlying definition.''

highlight {[}page 6{]}: ``Many theorists seem to believe that that
theory invariably precedes prac- tice. My own experience suggests that
the converse holds at least as often---that practice routinely leads
theory, and that it can take a long time for the theory to catch up.''

highlight {[}page 6{]}: ``H is collision resistant if it is computa-
tionally infeasible to find distinct strings X and X 0 such that H(X) =
H(X 0 ).''

highlight {[}page 6{]}: ``formalize the security goal we can define Adv
col (A) as the probability that adversary A out- H H puts distinct X and
X such that H(X) = H(X 0''

highlight {[}page 6{]}: ``nformally, we regard H as secure if ev- ery
``reasonable'' adversary A gets ``small'' advantage measure Adv col H "

highlight {[}page 6{]}: ``(A).''

highlight {[}page 6{]}: ``No matter what H may be, there will al- ways
be an efficient algorithm A that outputs a colli- sion for it---namely,
the efficient algorithm that knows a collision X, X 0 for H and outputs
it.''

highlight {[}page 7{]}: ``There will always exist an al- gorithm A that
prints out H-colliding 161-bit strings X, X 0 .''

highlight {[}page 7{]}: ``algorithm is as efficient as can be---just a
couple lines of code---and it gets advantage 1.''

highlight {[}page 7{]}: ``of course the collision-printing algorithm
exists; the difficulty is our inability to explicitly spec- ify it.''

highlight {[}page 7{]}: ``person---no living human being---who can write
a collision down.''

highlight {[}page 7{]}: ``mathematically rigorous treatment of hash
functions, certainly you can't base it on what human beings do or do not
know. N''

highlight {[}page 7{]}: ``definitions in mathematics have such a
character.''

highlight {[}page 7{]}: ``Instead of considering just one hash func-
tion, we {[}must{]} consider families of them, in order to make a
complexity-theoretic treat- ment possible''

highlight {[}page 7{]}: ``The beautiful idea of zero-knowledge proofs
was mo- tivated by simple protocols like the one I'll describe right
now. A prover would like to convince a verifier that a pair of graphs G
0 and G 1 are isomorphic. (Re- member that two graphs are isomorphic if
they are the same up to the naming of their vertices.)''

highlight {[}page 8{]}: ``And yet I think it's fair to say that
zero-knowledge, at least in the traditional form that I've described,
has had little impact on cryptographic practice. It somehow hasn't
mattered. I conclude that if a notion is elegant enough, real-world
applications may not be needed.''

highlight {[}page 8{]}: ``When you have a good notion, and you have a
good name for it as well, that is an unbeatable package. The phrase zero
knowledge manages to capture, in just these two words, a wonderful
paradox: how can something be knowledge if it is also zero? The name
itself already inspires the imagination. Good defi- nitions excite the
imagination and aspirations of a community.''
