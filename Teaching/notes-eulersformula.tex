\documentclass[10pt]{article}

\usepackage{enumitem}
\setlist{nosep}

\usepackage[none]{hyphenat}

\usepackage{amsmath}
\usepackage{amsthm}
\usepackage{amssymb}
\usepackage{tikz}

\usepackage{graphicx}
\usepackage{xcolor}

\usepackage{float}
\usepackage{caption}
\usepackage{stmaryrd}
\usepackage{framed}

\usepackage[parfill]{parskip}

\usepackage{fullpage}
\usepackage{multicol}

%VERSION
\newcounter{versionNumber}
\makeatletter

\AtEndDocument{%
	\immediate\write\@auxout{%
		\string\setcounter{versionNumber}{\number\value{versionNumber}}%
	}%
}

\makeatother
\AtBeginDocument{%
	\stepcounter{versionNumber}
}
%ENDVERSION

%HEADER
\usepackage{fancyhdr}
\pagestyle{fancy}
\usepackage{lastpage}
\setlength{\headheight}{15.2pt}
\setlength{\headsep}{0.4in}
\fancyhead{}
\rhead{\textit{Trigonometry}}
\lhead{K. M. Short}
\chead{\textsc{Notes:} \textit{Euler's Formula} }
\fancyfoot{}
\cfoot{Page \thepage/\pageref{LastPage}}
\lfoot{\textbf{Version:} \theversionNumber}
%ENDHEADER

\theoremstyle{definition}
\newtheorem*{defn}{Definition}
\newtheorem*{exmp}{Example}
\newtheorem*{cond}{Condition}
\newtheorem*{prob}{Problem}
\newtheorem*{exer}{Exercise}
\newtheorem*{que}{Question}
\newtheorem*{axi}{Axiom}
\newtheorem*{prop}{Property}
\newtheorem*{asu}{Assumption}
\newtheorem*{hypo}{Hypothesis}

\theoremstyle{plain}
\newtheorem*{thm}{Theorem}
\newtheorem*{lem}{Lemma}
\newtheorem*{cor}{Corollary}
\newtheorem*{pro}{Proposition}
\newtheorem*{conj}{Conjecture}
\newtheorem*{crit}{Criterion}
\newtheorem*{ass}{Assertion}

\theoremstyle{remark}
\newtheorem*{rem}{Remark}
\newtheorem*{note}{Note}
\newtheorem*{nota}{Notation}
\newtheorem*{cla}{Claim}
\newtheorem*{suu}{Summmary}
\newtheorem*{ack}{Acknowledgement}
\newtheorem*{case}{Case}
\newtheorem*{con}{Conclusion}

% Fonts
\newcommand{\mb}[1]{\mathbb{#1}}
\newcommand{\cb}[1]{\mathcal{#1}}
\newcommand{\fr}[1]{\mathfrak{#1}}
\newcommand{\rb}[1]{\mathrm{#1}}
\newcommand{\ib}[1]{\mathit{#1}}
\newcommand{\fb}[1]{\mathbf{#1}}
\newcommand{\se}[1]{\mathsf{#1}}
\newcommand{\tb}[1]{\mathtt{#1}}

% Integral
\newcommand{\intx}[2][x]{\int#2, \text{d}#1}
% Vector
\newcommand{\xvec}[2]{x_{#1},\ldots, x_{#2}}

\newcommand{\eul}[1]{e^{i#1} = \cos{(#1)} + i\sin{(#1)}}
\newcommand{\irt}{i = \sqrt{-1}}

\usepackage{url}
\usepackage{hyperref}
\hypersetup{
    colorlinks=true,
    linkcolor=blue,
    anchorcolor=black,
    citecolor=green,
    linktocpage=true,
    filecolor=magenta,      
    urlcolor=cyan,
    pdftitle={Notes},
    pdfauthor={K. M. Short},
    bookmarks=true,
    pdfpagemode=FullScreen,
    }


%%--------------------


\begin{document}

\section*{Review}

\textbf{Recall:} \\

$\cos{(\pi)} = (-1)$ and $\sin{(\pi)} = 0$

\textbf{Also remember:}

$\irt$ which means that we also have:

\[ i^{2} = (\sqrt{-1})^{2} = [(-1)^{\frac{1}{2}}]^{2} = (-1) \]

\[ i^{3} = (\sqrt{-1})^{3} = [(-1)^{\frac{1}{2}}]^{3} = (-1)^{\frac{3}{2}} = (-1)^{\frac{2}{2}}(-1)^{\frac{1}{2}} = (-1)(\sqrt{-1}) = -\sqrt{-1} = -i \]

\[ i^{4} = (\sqrt{-1})^{4} = [(-1)^{\frac{1}{2}}]^{4} = (-1)^{\frac{4}{2}} = (-1)^{2} = 1 \]

\[ i^{5} = (\sqrt{-1})^{5} = [(-1)^{\frac{1}{2}}]^{5} = (-1)^{\frac{5}{2}} = (-1)^{\frac{4}{2}}(-1)^{\frac{1}{2}} = (-1)^{2}(\sqrt{-1}) = i \]

\textbf{Notice the pattern above:} 

\[ i = \sqrt{-1} \]
\[ i^{2} = (-1) \]
\[ i^{3} = -i \]
\[ i^{4} = 1 \]
\[ i^5 = i \]

After raising $i$ to four consecutive powers we get back where we started from, namely back at $i$. We notice this fact, and we say that powers of $i$ are \textit{cyclic}.

This is going to come in handy so many times for so many different types of problems it will be somewhat creepy. But for now just keep it in the back of your head.

\section*{Want: To find any trigonometric function without resorting to memorization}

\textit{Why would we want to not memorize things?}

Believe it or not computer scientists and programmers are \textbf{not} the laziest major on any given college campus \footnote{think code reuse, attribute and use}. The laziness of programmers was inherited from (to a lesser degree) mathematics\footnote{Yes that is where CS came from, programmers can argue about it until the heat death of the universe but that doesn't make the fact that Turing was a mathematician any less true}. 

Mathematicians in general, \textbf{hate} to memorize anything. Mostly because the higher you go up in mathematics the more of a loosing battle that becomes. One would have a very hard time memorizing every formula in mathematics, it would be tedious and worse counter productive\footnote{So we don't memorize anything we don't have to, its hard enough to remember the meeting times of campus organizations with free food, which is a far higher priority}.


So what do we do? How will we get all these formulas without memorizing them? We will \textit{derive} them in our spare time. Then, when the next exam comes around and the professor uses that one identity you were really hoping she wouldn't there will be no need to panic.

Believe it or not, the above successive squaring of $i$ is a very simple (partial) derivation\footnote{derivation is used here in the sense of \textit{coming from} and does not refer to taking derivatives of functions (at least not in this particular set of notes)}.

In a sense when we are deriving, we are playing with our math in a very good way. We are familiarizing ourselves with \textit{why} something is true, instead of memorizing a few instances of \textit{when} it is true. If you know why something is true, then you can find every case where it is true, false, or both. But if you only know a handful of cases where you know something is true, you will work harder and not get much in return for all that effort.

\textit{So let's do math properly and be lazy, let's derive!}


\section*{Euler's Formula}

\begin{defn}{Complex Number \\}
A complex number $z \in \mathbb{C}$ is defined to be:

\[ z = x + iy \]
Where for all values $x, y \in \Re$, and $i \in \Im$ we have that $z \in \mathbb{C}$. 
\end{defn}

\medskip

Where we usually have an $x$-axis now we have an axis composed of real numbers $\Re$; and for our $y$-axis we have an axis made up of imaginary parts $\Im$. 


\begin{thm}{Euler's Formula \\} 

$ \eul{x} $
\end{thm}

\end{document}  