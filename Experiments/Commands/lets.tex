\documentclass{article}
\title{Defining Commands: let and glet}
\author{}
\date{}

\def\test#1#2{\emph{#1} \textbf{#2}}

def\td#1-#2-#3\endtd{\emph{#1} \textbf{#2} \texttt{#3}}


\begin{document}
\maketitle

The \texttt{let} and \texttt{glet} commands allow you to define commands locally and globally (respectively). 

Using the let commands you do not simply specify the number of arguments but the argument syntax, using the numbers 1 through 9 as parameters; you can also specify these commands without parameters.


\section*{test 2 parameters}

Definition test with 2 parameters using spaces as separators
\newline

\test{First}{Second}

\td First-Second-Third \endtd

\end{document}

