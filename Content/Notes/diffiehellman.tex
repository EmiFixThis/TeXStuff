\documentclass[10pt]{article}

\usepackage{amsmath,amsthm,amssymb}
\usepackage[parfill]{parskip}

\begin{document}

\section*{Diffie-Hellman and P-groups}
\hrule

\bigskip
So a group of prime power order is called a $p$-group.

$P$-groups are useful because they are cyclic groups, what this means is that:

If $G$ is a cyclic group and $g \in G$ is a generator for $G$ 
then $G = \langle g \rangle = \{ g^{n} : n \in \mathbb{Z} \}$.

That is if $G = \{ g^[0}, g^{1}, g^{2}, g^{3}, g^{4}, g^{5} \}$ 
is a group of order 6 
then we have that $g^{0} = g^{6}$, so $G$ is cyclic.

This is the same thing as saying that this group of order 6 is isomorphic to the set 
$\{0,1,2,3,4,5\}$ when we use addition $mod(6)$.

So in this example we could do this:
\[1+2 \equiv 3(mod 6) \]
and this is the same as writting $ g^{1} \cdot g^{2} = g^{3}$;

and $2 + 5 \equiv 1(mod 6)$ is the same as writing $g^{2} \cdot g^{5} = g^{7} = g^{1}$.

So we can use the isomorphism $\chi$
and define it as $\chi(g^{i}) = i$.

\textbf{It is possible to have infinite cyclic groups but Diffie-Hellman isn't concerned with those.}

So the Diffie-Hellman assumption considers a multiplicative cyclic group $G$ with order $q$ and a generator $g$.

Then the assumption says that if we are given $g^{a}$ and $g^{b}$,

where $a$ and $b$ are independently chosen from a random distribution of integers $mod(q)$,

then the exponential product $g^{ab}$ \emph{looks like} any other random element in $G$.


\subsection*{So using the discrete log assumption we have:}

\bigskip

\begin{itemize}
	\item A tuple $(g^a, g^b, g^{ab})$ such that $a,b \in \mathbb{Z}_q$ are randomly and independently chosen from $\mathbb{Z}_q$.
	\item and another tuple $(g^a, g^b, g^c)$ where $a,b,c \in \mathbb{Z}_q$ are again randomly and etc...
\end{itemize}

These are called \textbf{DDH tuples}.

So basically if you \emph{could} efficiently compute a discrete log from your group $G$ then DDH wouldn't hold in $G$. This is the key condition.

So this means that if you have $(g^a, g^b, z)$ 
you could decide if $z=g^{ab}$ by taking the discrete $log_{g}(g^a)$ 
and compare it to $(g^{b})^{a}$.

So DDH is considered a \emph{stronger} assumption than the discrete log assumption 
since you can find groups where finding DDH tuples is easy 
but by assumption the discrete log assumption is still hard. 

So since we require the group satisfy the DDH assumption 

this is harder than just the assumption that the discrete log is hard to compute in the group.

\subsection*{DDH is assumed to hold in the following groups:}

\begin{itemize}
	\item The subgroup of $kth$ residues mod a prime, such that $(p-1)/k$ is also a large prime. \emph{Schnorr group}
	\item A prime order elliptic curve over $E$ in the field $GF(p)$ where $p$ is a prime, provided $E$ has a large embedding degree.
	\item A Jacobian of a hyper-elliptic curve over the field $GF(p)$ with a prime number of reduced divisors where $p$ is a prime, and the Jacobian has a large embedding degree.
\end{itemize}

\section*{DDH DOES NOT HOLD IN:} 
\[G = \mathbb{Z}^{*}_{p} : p \in \mathbb{P}\]

\end{document}