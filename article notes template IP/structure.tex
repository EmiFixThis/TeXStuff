%%%%%%%%%%%%%%%%%%%%%%%%%%%%%%%%%%%%%%%%%
% Math Article Notes
% Structure Specification File
% Version 2.5 (04/29/2016)
%
% This file was modified from a template
% which was downloaded from:
% http://www.LaTeXTemplates.com
%
% Original Authors:
% Vel
% (vel@latextemplates.com)
% Christopher Eliot
% (christopher.eliot@hofstra.edu)
% Anthony Dardis
% (anthony.dardis@hofstra.edu)
%
% License:
% CC BY-NC-SA 3.0
% (http://creativecommons.org/licenses/by-nc-sa/3.0/)
%
% Current Version Author:
% Kris
% (kshort@iastate.edu)
% (https://www.github.com/krismshort)
%
%%%%%%%%%%%%%%%%%%%%%%%%%%%%%%%%%%%%%%%%%

%----------------------------------------------------------------------------------------
%	REQUIRED PACKAGES
%----------------------------------------------------------------------------------------
\usepackage[columnsep=1cm, left=0.25in, right=0.25in, margin=0.25in,bottom=.05in,top=.15in, bindingoffset=0.2in]{geometry}


\usepackage{hyperref}
\usepackage{url}
%----------------------------------------------------------------------------
% Fun with Todonotes!
%
% These packages are required for todonotes
\usepackage{ifthen}
\usepackage{keyval}
\usepackage{xcolor}
\usepackage{tikz}    % tikz loads graphicx automagically
\usepackage{calc}
%
% Now call todonotes after all that shit
% If you didn't call all that shit
% todonotes will just do it in the background
%
\usepackage[colorinlistoftodos]{todonotes}
%
% Fun Now!
%
% \todo{This is a marginal todo note}
% \todo[fancyline]{This is a marginal note with a curvy line.}
% \todo[noline]{A margin todo with no connecting line.}
%
% Missing figure todonotes
% \missingfigure{This inserts a big dummy figure to remind you to put a figure in}
% \missingfigure[figwidth=6cm]{Testing a long text string}
% \missingfigure[figheight=6cm]{Testing a long text string}

% \listoftodos [No arguments, just prints out a list of all the todos in the document]
% \todototoc [No arguments, makes an entry in the table of contents for the list of todos section, MUST CALL BEFORE LISTOFTODOS COMMAND]
%
% You can use different font sizes
%
% \todo[size=\Large]{A margin note with the Large font size.}
% \todo[size=\tiny]{A margin note with tiny font.}
% \todo[size=\tiny, inline]{A inline todo with tiny font.}
%
% You can make inline notes
% \todo[inline]{An inline todo note.}
% You can insert the notes as inline notes inside the figure environments.}
% \caption{A Caption!}
% \todo[inline]{Fix this stupid caption}
% \end{figure}
%
% You can make a caption for a Huge todonote
%\todo[caption={Short note}]{A very long and tedious note that cannot be on one line in the list of todos.}
%
% You can change the border, background, and text color
% Example using all available colors except the defaults
% \todo[linecolor=green!70!white, backgroundcolor=blue!20!white,bordercolor=red]{Anything but default colors}
% \todo[color=green!40]{And a green note}
%
% color changing options are:
% color : color of note
% backgroundcolor
% linecolor
% bordercolor
% You can define notes for specific tasks, priority, etc...
% Here are some custom tasks I have defined:
%
% Reminder to get an answer to a question
\newcommand{\answerquestion}[1]{\todo[backgroundcolor=green!40, linecolor=green!50, bordercolor=blue!40, ]{#1}}
% This is a really fucking important todonote
\newcommand{\fuckingimportant}[1]{\todo[size=\Huge, backgroundcolor=red!50, linecolor=black, bordercolor=black, author=IMPORTANT]{#1}}
%
% Insert reference
\newcommand{\insertref}[1]{\todo[backgroundcolor=blue!40, linecolor=black!40, bordercolor=black!40]{#1}}
% Insert list of papers cited by author
\newcommand{\insertcited}[1]{\todo[color=blue!20, noline]{#1}}
%
%
% You can make a caption for a Huge todonote
%\todo[caption={Short note}]{A very long and tedious note that cannot be on one line in the list of todos.}
% \todo[prepend, caption={Short note with prepend}]{A very long and tedious note that cannot be on one line in the list of todos.}.
% \todo[noprepend, caption={Short note with noprepend}]{A very long and tedious note that cannot be on one line in the list of todos.}
% You can assign and sign todonotes
% \todo[author=Xavier]{Testing author option.}
%----------------------------------------------------------------------------

\setlength{\voffset}{-0.75in}
\setlength{\headsep}{0pt}
\usepackage{fullpage}


\usepackage{hyperref}
\usepackage{url}
%----------------------------------------------------------------------------

\usepackage[english]{babel} % Use english by default


%----------------------------------------------------------------------------
% AMS Stuff
%
\usepackage{amsmath}
\usepackage{amssymb}
\usepackage{amsthm}
%
% AMS Theorem Package Custom Definitions
%
\theoremstyle{definition}
\newtheorem*{defn}{Definition. \\}
\newtheorem{que}{Question: \\}

\theoremstyle{theorem}
\newtheorem*{thm}{Theorem}{}

\theoremstyle{remark}
\newtheorem{rem}{Remark: \\}
\newtheorem*{nt}{Note: \\}
\newtheorem{ans}{Answer: \\}


%----------------------------------------------------------------------------
% Figure Stuff
%
\usepackage{float}
% Favorite Usage
% PUT THE DAMN FIGURE HERE NO MATTER WHAT THE COMPILER THINKS
% \begin{figure}[H]
%
%----------------------------------------------------------------------------
% Back Matter Stuff
%
\usepackage{amsrefs}
\bibliographystyle{amsalpha}
%
% Some very special packages (that sometimes just refuse to work no matter what the fuck you do)
%
\usepackage{gloss}
\usepackage{concepts}

%----------------------------------------------------------------------------
% Formatting and Misc
%
\usepackage{verbatim}

\usepackage{stmaryrd}    % For contradiction Symbol: \lightning

%
%----------------------------------------------------------------------------
% Lists
%
\usepackage{enumerate}

% Ways to number lists

% \begin{enumerate}[(a)]
% Or any of these ways
% [a]
% [a.]
% [(A)]
% [A.]
% [(i)]
% [(I)]
% [i.]
% [(i.)]
% change separation between items in lists
% DOES NOT SEEM TO WANT TO WORK
%\begin{itemize}\itemsep2pt
%----------------------------------------------------------------------------


%----------------------------------------------------------------------------
%
% Template Revision Number
%
\newcommand{\revisionnumber}[1]{\renewcommand{\revisionnumber}{#1}}
%----------------------------------------------------------------------------
% DOCUMENT STUFF
%----------------------------------------------------------------------------
%
% Question Information
%
%
% Answer Information
%
\newcommand{\answerfound}[1]{\renewcommand{\answerfound}{#1}}
\newcommand{\dateanswerfound}[1]{\renewcommand{\dateanswerfound}{#1}}
\newcommand{\answerauthor}[1]{\renewcommand{\answerauthor}{#1}}
%
% Article Information
%
%% Article Title
\newcommand{\articletitle}[1]{\renewcommand{\articletitle}{#1}}
%% Article Citation Key
\newcommand{\articlecitation}[1]{\renewcommand{\articlecitation}{#1}}
%
%% DocumentTitle Format (Not Used)
\newcommand{\doctitle}{\articlecitation\ --- ``\articletitle''}
%
%% Date You Started Analysis
\newcommand{\datenotesstarted}[1]{\renewcommand{\datenotesstarted}{#1}}
%% Store the date
\newcommand{\docdate}[1]{\renewcommand{\docdate}{#1}}
%
%% Store Author of Article
\newcommand{\docauthor}[1]{\renewcommand{\docauthor}{#1}}
%
% Store AMS Info
%
%% AMS Class
\newcommand{\amsclass}[1]{\renewcommand{\amsclass}{#1}}
%% AMS Review
\newcommand{\amsreview}[1]{\renewcommand{\amsreview}{#1}}
%%% NOTE ADD:
%\url{the web address of the review}
%
%% Keywords
\newcommand{\keywords}[1]{\renewcommand{\keywords}{#1}}
%
%
% Header and Footer
\usepackage{fancyhdr}
\pagestyle{fancy}
\usepackage{lastpage}

\setlength{\headheight}{25.2pt}
\setlength{\headsep}{0.4in}
\renewcommand{\headrulewidth}{1pt} 
\renewcommand{\footrulewidth}{1pt}

\fancyhf{}
\lhead{\textbf{\articlecitation}}
\chead{\textbf{Math Article Notes}}
\rhead{\textbf{\amsclass}}
\cfoot{\articletitle}}
\rfoot{Page \thepage/\pageref{LastPage}}
\lfoot{\datenotesstarted}


% Unused in this format
% Define a command for the structure of the document title
%\newcommand{\printtitle}{
%\begin{center}
%\textbf{\Large{\doctitle}}

%\docdate

%\docauthor
%\end{center}
%}

%----------------------------------------------------------------------------------------
%	STRUCTURE MODIFICATIONS
%----------------------------------------------------------------------------------------
\usepackage[parfill]{parskip}
%\setlength{\parskip}{2pt} % Slightly increase spacing between paragraphs

% Uncomment to center section titles
%\usepackage{sectsty}
%\sectionfont{\centering}

% Uncomment for Roman numerals for section numbers
\renewcommand\thesection{\Roman{section}}
