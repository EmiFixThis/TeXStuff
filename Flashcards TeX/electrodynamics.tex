% Electrodynamics Flashcards
% copyright 2006 Jason Underdown

\documentclass[avery5371,grid]{flashcards}

\cardfrontstyle[\large\slshape]{headings}
\cardbackstyle{empty}

\usepackage{amsmath}

\def\lim{\mathop{\rm lim}}
\newcommand{\xhat}[0]{\mathbf{\hat x}}
\newcommand{\yhat}[0]{\mathbf{\hat y}}
\newcommand{\zhat}[0]{\mathbf{\hat z}}

\begin{document}

\cardfrontfoot{Electrodynamics}

\begin{flashcard}[Copyright \& License]{Copyright \copyright \, 2007 Jason Underdown \\
Some rights reserved.}
\vspace*{\stretch{1}}
These flashcards and the accompanying \LaTeX \, source code are licensed
under a Creative Commons Attribution--NonCommercial--ShareAlike 2.5 License.  
For more information, see creativecommons.org.  You can contact the author at:
\begin{center}
\begin{small}\tt jasonu [remove-this] at physics dot utah dot edu\end{small}
\end{center}
\vspace*{\stretch{1}}
\end{flashcard}

\begin{flashcard}[Definition]{gradient}
\vspace*{\stretch{1}}
The gradient $\nabla T$ points in the direction of maximum increase of the 
function $T$.

\medskip
\begin{displaymath}
\nabla T \equiv \xhat \frac{\partial T}{\partial x} +
\yhat \frac{\partial T}{\partial y} +
\zhat \frac{\partial T}{\partial z}
\end{displaymath}

\medskip
The magnitude $|\nabla T|$ is the slope along this direction.
\vspace*{\stretch{1}}
\end{flashcard}

\begin{flashcard}[Definition]{the vector operator $\nabla$}
\vspace*{\stretch{1}}
\begin{displaymath}
\nabla \equiv \xhat \frac{\partial}{\partial x} +
\yhat \frac{\partial}{\partial y} +
\zhat \frac{\partial}{\partial z}
\end{displaymath}
\vspace*{\stretch{1}}
\end{flashcard}

\begin{flashcard}[Definition]{divergence}
\vspace*{\stretch{1}}
\begin{eqnarray*}
\nabla \cdot \mathbf v &=& (\xhat \frac{\partial}{\partial x} +
\yhat \frac{\partial}{\partial y} +
\zhat \frac{\partial}{\partial z}) \cdot
(v_x \xhat + v_y \yhat + v_z \zhat)\\
&=& \frac{\partial v_x}{\partial x} +
\frac{\partial v_y}{\partial y} +
\frac{\partial v_z}{\partial z}
\end{eqnarray*}
The \textit{divergence} is a measure of how much the vector function
$\mathbf v$ spreads out from the point in question.
\vspace*{\stretch{1}}
\end{flashcard}

\begin{flashcard}[Definition]{curl}
\vspace*{\stretch{1}}
\begin{displaymath}
\nabla \times \mathbf v  =
\begin{vmatrix}
\xhat & \yhat & \zhat\\
\frac{\partial}{\partial x} &
\frac{\partial}{\partial y} &
\frac{\partial}{\partial z}\\
v_x & v_y & v_z
\end{vmatrix}
\end{displaymath}
The \textit{curl} is a measure of how much the vector field ``curls around''
the point in question.
\vspace*{\stretch{1}}
\end{flashcard}

\begin{flashcard}[Definition]{5 species of second derivatives}
\vspace*{\stretch{1}}
\begin{small}
By applying $\nabla$ twice we can construct five species of second derivatives.
\begin{enumerate}
\item divergence of a gradient $\nabla \cdot (\nabla T) = \nabla^2$ (Laplacian)
\item curl of a gradient $\nabla \times (\nabla T) = 0$ (always)
\item gradient of a divergence $\nabla(\nabla \cdot \mathbf v)$ (seldom occurs)
\item divergence of a curl $\nabla \cdot (\nabla \times \mathbf v) = 0$ (always)
\item curl of a curl $\nabla \times (\nabla \times \mathbf v) = 
\nabla(\nabla \cdot \mathbf v) - \nabla^2 \mathbf v$
\end{enumerate}
\end{small}
\vspace*{\stretch{1}}
\end{flashcard}

\begin{flashcard}[Theorem]{curl--less or irrotational fields}
\vspace*{\stretch{1}}
For a given vector field $\mathbf F$ the following statements are equivalent, i.e. each implies the others.
\begin{enumerate}
\item $\nabla \times \mathbf F = 0$ everywhere
\item $\int_{\mathbf a}^{\mathbf b} \mathbf F \cdot d \mathbf l$ is path independent
\item $\oint \mathbf F \cdot d \mathbf l = 0$ on any closed loop
\item $\mathbf F = -\nabla V$ for some scalar potential $V$
\end{enumerate}
\vspace*{\stretch{1}}
\end{flashcard}

\begin{flashcard}[Theorem]{divergence--less or solenoidal fields}
\vspace*{\stretch{1}}
For a given vector field $\mathbf F$ the following statements are equivalent, i.e. each implies the others.
\begin{enumerate}
\item $\nabla \cdot \mathbf F = 0$ everywhere
\item $\int \mathbf F \cdot d \mathbf a$ is independent of surface
\item $\oint \mathbf F \cdot d \mathbf a = 0$ over any closed surface
\item $\mathbf F = \nabla \times \mathbf A$ for some vector potential $\mathbf A$
\end{enumerate}
\vspace*{\stretch{1}}
\end{flashcard}

\begin{flashcard}[Theorem]{gradient theorem}
\vspace*{\stretch{1}}
\begin{displaymath}
\int_{\mathbf{a}}^{\mathbf{b}} (\nabla f) \cdot d \mathbf{l} = 
f(\mathbf{b}) - f(\mathbf{a})
\end{displaymath}
\vspace*{\stretch{1}}
\end{flashcard}

\begin{flashcard}[Theorem]{Green's theorem}
\vspace*{\stretch{1}}
\begin{displaymath}
\int (\nabla \cdot \mathbf{A}) dV = \oint \mathbf A \cdot d \mathbf a
\end{displaymath}
\vspace*{\stretch{1}}
\end{flashcard}

\begin{flashcard}[Theorem]{Stokes' theorem}
\vspace*{\stretch{1}}
\begin{displaymath}
\int (\nabla \times \mathbf A) \cdot d\mathbf a = \oint \mathbf A \cdot d \mathbf l
\end{displaymath}
\vspace*{\stretch{1}}
\end{flashcard}




\end{document} 
